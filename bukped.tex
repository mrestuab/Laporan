\documentclass[12pt, a4paper]{report}

% Required Packages
\usepackage[utf8]{inputenc}
\usepackage[T1]{fontenc}
\usepackage{mathptmx}   % Times New Roman for text
\usepackage{courier}    % Courier for code
\usepackage{helvet}     % Helvetica for sans-serif
\usepackage{amsmath}
% \usepackage[indonesian]{babel} % DISABLED to prevent errors
\usepackage{geometry}
\usepackage{setspace}
\usepackage{titlesec}
\usepackage{graphicx}
\usepackage{xcolor}
\usepackage{float}
\usepackage{hyperref}
% Disable colored boxes and annotate borders explicitly
\hypersetup{
  colorlinks=true,
  linkcolor=black,
  citecolor=black,
  urlcolor=black,
  pdfborder={0 0 0},
  pdfborderstyle={/S/U/W 0}
}
\usepackage{booktabs}
\usepackage{array}
\usepackage{caption}
\usepackage{enumitem}
\usepackage{parskip}
\usepackage{fancyhdr}
\usepackage{tocloft}
\usepackage{tabularx}
\usepackage{longtable}
\usepackage{ragged2e}
\usepackage{xstring}

% Manual Localization (Replaces babel functionality)
\renewcommand{\contentsname}{DAFTAR ISI}
\renewcommand{\listfigurename}{DAFTAR GAMBAR}
\renewcommand{\listtablename}{DAFTAR TABEL}
\renewcommand{\bibname}{DAFTAR PUSTAKA}
\renewcommand{\chaptername}{BAB}
\renewcommand{\figurename}{Gambar}
\renewcommand{\tablename}{Tabel}

% Page Geometry
\geometry{
    left=4cm,
    top=3cm,
    right=3cm,
    bottom=3cm
}

% Spacing
\onehalfspacing
\setlength\emergencystretch{2em}

% Center chapter titles and make chapter label same size as title
\titleformat{\chapter}[display]{\normalfont\bfseries\centering}{\Huge\chaptername\ \thechapter}{12pt}{\Huge}

% Use tocloft to center table of contents / list of figures / list of tables titles
\renewcommand{\contentsname}{DAFTAR ISI}
\renewcommand{\listfigurename}{DAFTAR GAMBAR}
\renewcommand{\listtablename}{DAFTAR TABEL}

% Center the LOF / LOT titles using cft title font + after-title filler (omit TOC cft overrides to avoid conflicts)
\renewcommand{\cftloftitlefont}{\hfill\large\bfseries}
\renewcommand{\cftafterloftitle}{\hfill}
\renewcommand{\cftlottitlefont}{\hfill\large\bfseries}
\renewcommand{\cftafterlottitle}{\hfill}

% Tidy TOC: set indents and spacing for nicer appearance
\cftsetindents{chapter}{0em}{3.0em}
\cftsetindents{section}{1.8em}{3.0em}
\cftsetindents{subsection}{4.2em}{3.0em}
\setlength{\cftbeforechapskip}{0.6ex}
\setlength{\cftbeforesecskip}{0.2ex}
\setlength{\cftbeforesubsecskip}{0.1ex}

% Replace default list printing by low-level calls so we control the title placement precisely
\makeatletter
\renewcommand{\tableofcontents}{%
  \clearpage
  % use empty pagestyle for entire TOC (covers multiple pages)
  \pagestyle{empty}
  \thispagestyle{empty}
  % center across the full page width (ignore left/right text margins)
  \noindent\makebox[\paperwidth]{\hfill\large\bfseries DAFTAR ISI\hfill}\par
  \vspace{0.5cm}
  \@starttoc{tableofcontents}
  \clearpage
  % restore document pagestyle (fancy if used)
  \pagestyle{fancy}
}

\renewcommand{\listoffigures}{%
  \clearpage
  \pagestyle{empty}
  \thispagestyle{empty}
  \begin{center}\large\bfseries DAFTAR GAMBAR\end{center}
  \vspace{0.5cm}
  \@starttoc{lof}
  \clearpage
  \pagestyle{fancy}
}
% \listoftables disabled (removed by author request)
\renewcommand{\listoftables}{\relax}
\makeatother

\begin{document}

% --- FRONT MATTER (cover page) ---
\begin{titlepage}
\thispagestyle{empty}
\centering
\vspace*{1.6cm}
{\bfseries\Large NYUMBANGIN: PLATFORM DONASI DIGITAL}\\[0.8cm]
{\bfseries\normalsize LAPORAN PROYEK III}\\[0.8cm]
{\small Diajukan untuk Memenuhi Kelulusan Matakuliah}\\[0.4em]
{\small Proyek 3 pada Program Studi DIV Teknik}\\[0.2em]
{\small Informatika}\\[1.0cm]

% University logo (centered & proportionate)
\includegraphics[width=0.6\textwidth,keepaspectratio]{images/ulbi.png}\\[0.9cm]

{\bfseries DISUSUN OLEH :}\\[0.4cm]
{\small\textbf{714230027\,-\,Muhamad Haekal Syukur}}\\[0.2em]
{\small\textbf{714230060\,-\,Muhammad Ferdy Leoza}}\\[1.0cm]

{\small\bfseries PROGRAM STUDI DIV TEKNIK INFORMATIKA}\\
{\small\bfseries FAKULTAS SEKOLAH VOKASI}\\
{\small\bfseries UNIVERSITAS LOGISTIK \& BISNIS INTERNASIONAL}\\[0.6em]
{\small\bfseries BANDUNG}\\
{\small\bfseries 2026}
\vfill
\end{titlepage}
\clearpage
% --- end FRONT MATTER ---

% --- LEMBAR PENGESAHAN (single page) ---
\begin{titlepage}
\centering
\vspace*{1cm}
{\bfseries\large LEMBAR PENGESAHAN}\\[1ex]
{\bfseries\large NYUMBANGIN: PLATFORM DONASI DIGITAL}\\[1.2cm]

\begin{center}
  Laporan Proyek 3 ini telah diperiksa, disetujui, dan disidangkan\\[0.5em]
  Di Bandung, \quad Oleh:
\end{center}
\vspace{0.8cm}

% signature block: examiners / pembimbing / koordinator
\begin{center}
\begin{tabular}{ >{\centering\arraybackslash}p{7cm} >{\centering\arraybackslash}p{7cm} }
  Penguji Pendamping & Penguji Utama \\\vspace{1.2cm} & \\
  \underline{\textbf{Rolly Maulana Awangga, S.T., M.T.}} & \underline{\textbf{Dr. Syafrial Fachri Pane, S.T., M.T.}} \\
  NIK: 117.86.219 & NIK: 117.88.233 \\
\end{tabular}
\end{center}

\vspace{0.9cm}
\begin{center}
\begin{tabular}{ >{\centering\arraybackslash}p{7cm} >{\centering\arraybackslash}p{7cm} }
  Pembimbing & Koordinator Proyek 3 \\\vspace{1.2cm} & \\
  \underline{\textbf{Rolly Maulana Awangga, S.T., M.T.}} & \underline{\textbf{Roni Habibi, S.Kom., M.T., SFPC}} \\
  NIK: 117.86.219 & NIK: 103.78.069 \\
\end{tabular}
\end{center}

\vfill
\begin{center}
    Menyetujui,\\
    Ketua Program Studi D-IV Teknik Informatika,\\[1.2cm]
    \textbf{Roni Andarsyah, S.T., M.Kom}\\
    NIK: 115.88.193
\end{center}
\vspace*{0.5cm}
\end{titlepage}
% --- end LEMBAR PENGESAHAN ---

% --- SURAT PERNYATAAN 1 (single page) ---
\begin{titlepage}
\centering
\vspace*{1cm}
{\bfseries\large SURAT PERNYATAAN BEBAS PLAGIARISME}\\[1ex]
\begin{tabular}{ll}
Nama & : Muhamad Haekal Syukur \\
NPM & : 714230027 \\
Program Studi & : DIV Teknik Informatika \\
Judul & : NYUMBANGIN: PLATFORM DONASI DIGITAL
\end{tabular}
\vspace{0.5cm}

Menyatakan bahwa:
\begin{enumerate}
    \item Proyek pemrograman aplikasi (PROYEK 3) ini adalah karya asli yang belum pernah diajukan untuk memenuhi kelulusan pada program studi DIV Teknik Informatika di Universitas Logistik \& Bisnis Internasional maupun di perguruan tinggi lainnya.
    \item Proyek ini merupakan hasil pemikiran, rumusan, dan penelitian saya sendiri, tanpa adanya bantuan dari pihak lain, kecuali arahan yang diberikan oleh pembimbing.
    \item Dalam proyek ini, tidak terdapat karya atau pendapat yang ditulis atau dipublikasikan oleh orang lain, kecuali jika telah dicantumkan secara tertulis sebagai acuan dalam naskah.
    \item Saya menyatakan bahwa pernyataan ini dibuat dengan sesungguhnya.
\end{enumerate}

\vfill
\begin{flushright}
    Bandung, Januari 2026\\
    Yang membuat pernyataan,\\[2cm]
    \textbf{Muhamad Haekal Syukur}\\
    NPM : 714230027
\end{flushright}
\end{titlepage}
% --- end SURAT PERNYATAAN 1 ---

% --- SURAT PERNYATAAN 2 (single page) ---
\begin{titlepage}
\centering
\vspace*{1cm}
{\bfseries\large SURAT PERNYATAAN BEBAS PLAGIARISME}\\[1ex]
\begin{tabular}{ll}
Nama & : Muhammad Ferdy Leoza \\
NPM & : 714230060 \\
Program Studi & : DIV Teknik Informatika \\
Judul & : NYUMBANGIN: PLATFORM DONASI DIGITAL
\end{tabular}
\vspace{0.5cm}

Menyatakan bahwa:
\begin{enumerate}
    \item Proyek pemrograman aplikasi (PROYEK 3) ini adalah karya asli yang belum pernah diajukan untuk memenuhi kelulusan pada program studi DIV Teknik Informatika di Universitas Logistik \& Bisnis Internasional maupun di perguruan tinggi lainnya.
    \item Proyek ini merupakan hasil pemikiran, rumusan, dan penelitian saya sendiri, tanpa adanya bantuan dari pihak lain, kecuali arahan yang diberikan oleh pembimbing.
    \item Dalam proyek ini, tidak terdapat karya atau pendapat yang ditulis atau dipublikasikan oleh orang lain, kecuali jika telah dicantumkan secara tertulis sebagai acuan dalam naskah.
    \item Saya menyatakan bahwa pernyataan ini dibuat dengan sesungguhnya.
\end{enumerate}

\vfill
\begin{flushright}
    Bandung, Januari 2026\\
    Yang membuat pernyataan,\\[2cm]
    \textbf{Muhammad Ferdy Leoza}\\
    NPM : 714230060
\end{flushright}
\end{titlepage}
% --- end SURAT PERNYATAAN 2 ---

\chapter*{ABSTRAK}
Aplikasi donasi digital kini banyak digunakan oleh kreator konten untuk memudahkan dukungan dari para pendukung. Laporan ini membahas pengembangan platform donasi dengan fitur utama seperti pengiriman donasi, notifikasi real-time melalui overlay, leaderboard pendukung, serta mekanisme pencairan dana bagi kreator. Pengembangan dilakukan melalui analisis kebutuhan, perancangan sistem, dan implementasi fitur sesuai alur donasi hingga pencairan dana. Hasil pengujian menunjukkan bahwa aplikasi dapat memproses donasi dengan baik, menampilkan notifikasi secara langsung dan menyediakan proses pencairan dana yang terstruktur untuk kreator. Secara keseluruhan, aplikasi yang dibangun telah memenuhi tujuan utama, yaitu menyediakan sarana donasi yang fungsional dan mudah digunakan, meskipun masih terdapat ruang pengembangan lebih lanjut untuk meningkatkan stabilitas dan cakupan fitur.

\vspace{0.5cm}
\textbf{Kata Kunci:} donasi digital, aplikasi web, kreator, notifikasi overlay, payout.

\chapter*{ABSTRACT}
\textit{Digital donations applications are now widely used by content creators to make easier for supporters to contribute. This report covers the development of a donation platform with key features such as donation delivery, real-time notifications via overlays, supporter leaderboards, and fund disbursement for creators. The development process included needs analysis, system design, and feature implementation following the donations flow up to payout. Testing show the applications can process donations well, display notifications instantly, and provide a structured fund disbursement process for creators. Overall, the applications that was built has met its main objective, which is to provide a functional and easy-to-use donation tool, although there is still room for further development to improve stability and feature coverage.}

\vspace{0.5cm}
\textbf{Keywords:} \textit{digital donations, web applications, creator, overlay notifications, payouts.}

\clearpage
\tableofcontents
\clearpage
\listoffigures
\clearpage% \listoftables removed as requested


% --- MAIN CONTENT ---
% Wrapper file so bukped.tex (which includes chapters/introduction) finds Bab 1
% This simply inputs the existing bab1.tex
\section{Latar Belakang}

Keterbatasan akses terhadap alat bantu medis di Indonesia masih menjadi masalah kesehatan masyarakat yang signifikan. Banyak penyandang disabilitas dan pasien dengan kebutuhan medis khusus mengalami kesulitan mendapatkan alat bantu seperti kursi roda, tabung oksigen, dan alat rehabilitasi lainnya karena faktor ekonomi dan keterbatasan ketersediaan alat di fasilitas kesehatan. Akses yang terbatas ini tidak hanya memperburuk kondisi kesehatan pengguna tetapi juga berdampak pada tingkat kecacatan dan risiko kematian akibat tidak terpenuhinya kebutuhan dasar medis \cite{pudjiastuti2022}.

Banyak fasilitas kesehatan, terutama rumah sakit umum, sering kali kehabisan stok alat bantu medis yang dibutuhkan oleh pasien, sementara kemampuan masyarakat untuk membeli alat tersebut secara mandiri masih rendah. Hal ini menunjukkan perlunya solusi alternatif agar alat bantu medis dapat diakses oleh masyarakat tanpa harus membeli dengan biaya tinggi \cite{fernandes2022}.

Dalam konteks tersebut, pengembangan sistem peminjaman alat bantu medis menjadi solusi inovatif untuk mengatasi keterbatasan akses tersebut. Sistem ini memungkinkan masyarakat memperoleh alat bantu medis secara temporer melalui mekanisme peminjaman, terutama bagi mereka yang belum mampu membeli alat tersebut atau ketika stok di rumah sakit habis. Dengan adanya sistem ini, diharapkan ketersediaan alat bantu medis bagi masyarakat dapat meningkat serta risiko komplikasi kesehatan akibat keterbatasan alat dapat dikurangi.

\section{Deskripsi Aplikasi}

Aplikasi \textit{MedisLink} merupakan platform berbasis web yang dirancang untuk membantu masyarakat memperoleh akses alat bantu medis secara lebih mudah, cepat, dan efisien melalui sistem peminjaman. Aplikasi ini menjadi solusi bagi individu yang membutuhkan alat bantu medis seperti kursi roda, tongkat kruk, \textit{walker}, atau alat pernapasan namun terkendala biaya atau ketersediaan alat di fasilitas kesehatan.

Melalui MedisLink, pengguna dapat melihat ketersediaan alat, melakukan peminjaman, melacak status pinjaman, serta menerima notifikasi pengembalian secara otomatis. Selain itu, sistem juga mempermudah pengelolaan inventaris bagi admin sehingga pemantauan stok dapat dilakukan secara transparan dan akurat.

Fitur utama aplikasi MedisLink meliputi:

\begin{enumerate}
    \item \textbf{Peminjaman Alat Bantu Medis} \\
    Pengguna dapat memilih jenis alat yang tersedia, melihat detail alat, dan melakukan peminjaman secara daring. Setiap transaksi tersimpan dalam sistem sehingga status peminjaman dapat dipantau.

    \item \textbf{Pengelolaan Inventaris Alat} \\
    Admin dapat mengelola stok alat, menambahkan alat baru, memperbarui kondisi alat, serta mencatat riwayat keluar-masuk alat melalui fitur \textit{Inventory Log}.

    \item \textbf{Notifikasi Pengembalian} \\
    Sistem mengirimkan notifikasi kepada pengguna untuk mengingatkan jadwal pengembalian alat agar dapat digunakan kembali oleh peminjam lainnya.
\end{enumerate}

\section{Identifikasi Masalah}

Berdasarkan latar belakang tersebut, permasalahan yang diidentifikasi adalah:

\begin{enumerate}
    \item Bagaimana membangun sistem peminjaman alat bantu medis yang memudahkan masyarakat mengakses alat tanpa harus membelinya?
    \item Bagaimana sistem dapat membantu mengatasi keterbatasan stok alat bantu medis di rumah sakit atau fasilitas kesehatan?
    \item Bagaimana memfasilitasi masyarakat yang memiliki alat bantu medis tidak terpakai agar dapat mendonasikannya secara tepat sasaran?
\end{enumerate}

\section{Tujuan}

Tujuan pengembangan sistem ini adalah menyediakan solusi berupa sistem peminjaman alat bantu medis yang dapat memperluas akses masyarakat terhadap alat kesehatan.

Tujuan khusus proyek ini adalah:

\begin{enumerate}
    \item Merancang dan membangun platform peminjaman alat bantu medis berbasis web.
    \item Mengembangkan sistem manajemen inventaris untuk melacak perputaran dan ketersediaan alat.
    \item Menyediakan fitur donasi yang memfasilitasi penyaluran alat bantu medis yang tidak terpakai.
\end{enumerate}

\section{Ruang Lingkup}

\textbf{Pengembangan Teknologi:}

Dokumentasi ini membahas proses pengembangan sistem peminjaman alat bantu medis berbasis web dengan menggunakan teknologi seperti Golang (Fiber) sebagai \textit{backend}, React.js sebagai \textit{frontend}, dan MongoDB sebagai basis data. Sistem dilengkapi dengan autentikasi JWT untuk menjaga keamanan akses pengguna serta struktur API yang efisien, yaitu mampu memberikan respons dengan cepat, mengirim data seperlunya, dan memiliki pembagian fungsi yang jelas pada setiap layanan \cite{fiber}. Selain itu, sistem juga menyediakan fitur notifikasi pengembalian alat yang berfungsi untuk mengingatkan pengguna mengenai batas waktu pengembalian alat yang dipinjam. Dokumentasi ini menjelaskan proses pengembangan sistem secara menyeluruh, mulai dari perancangan arsitektur, desain database, hingga cara kerja setiap fitur dalam aplikasi.

\textbf{Peningkatan Akses dan Efisiensi Layanan:}

Melalui pengembangan sistem MedisLink, dokumentasi ini membahas bagaimana aplikasi dapat membantu meningkatkan akses masyarakat terhadap alat bantu medis. Dengan adanya fitur peminjaman online, pengguna tidak perlu datang langsung ke fasilitas kesehatan atau tempat penyedia alat, sehingga proses peminjaman menjadi lebih cepat dan efisien. Sistem ini juga dilengkapi dengan tampilan kategori alat, ulasan pengguna, serta pengelolaan inventaris yang terstruktur.

\chapter{Landasan Teori}

\section{Platform Donasi Digital}

Platform donasi digital merupakan pengembangan dari teknologi platform berbasis internet yang memfasilitasi interaksi antara berbagai pihak untuk tujuan sosial dan filantropi. Platform digital didefinisikan sebagai seperangkat komponen teknologi yang menyediakan fungsi inti bagi suatu sistem dan menjadi fondasi bagi pengembangan layanan pelengkap di atasnya \cite{shneor2020,zhou2025}. Secara konseptual, platform ini beroperasi sebagai \textit{two-sided market} yang mempertemukan kelompok pengguna berbeda namun saling bergantung, seperti donatur dan penerima manfaat, di mana nilai platform tercipta dari interaksi antar pengguna tersebut \cite{jullien2021}.

Dalam konteks filantropi di Indonesia, platform digital digunakan sebagai alternatif lembaga amil konvensional dengan menawarkan kemudahan akses, transparansi, dan kecepatan distribusi dana untuk zakat, infaq, sedekah, dan wakaf \cite{febriandika2024,hidayatullah2022}. Perkembangan ini sejalan dengan meningkatnya kepercayaan masyarakat terhadap sistem donasi daring serta pergeseran perilaku filantropi ke media digital \cite{indriyani2024}.

Dalam lingkup penggalangan dana massal, platform donation-based crowdfunding didefinisikan secara spesifik sebagai perantara teknologi yang digunakan oleh penggalang dana untuk mencocokkan donasi dengan tujuan para donatur. Tujuan yang dipertemukan dalam platform ini tidak bersifat moneter, melainkan pemenuhan kebutuhan psikologis donatur \cite{fadzirul2020}. Dalam ekosistem ini, penyedia situs web berperan menyediakan layanan web dan membangun sistem yang mendukung pemilik proyek untuk mempresentasikan kampanye mereka kepada calon pendukung guna menggalang donasi \cite{fadzirul2020,sirisawat2022}.

\section{Arsitektur Aplikasi Web Modern}

Bagian ini membahas konsep dasar arsitektur aplikasi web modern yang menjadi landasan dalam memahami cara kerja sistem berbasis web \cite{fraihat2022}. Pemahaman mengenai pola komunikasi serta pembagian lapisan dalam aplikasi diperlukan untuk menjelaskan bagaimana komponen penyusun sistem saling berinteraksi dan menjalankan fungsinya. Oleh karena itu, pembahasan berikut difokuskan pada model client--server dan arsitektur three-tier/N-tier sebagai struktur arsitektural yang umum digunakan dalam pengembangan aplikasi web masa kini.

\subsection{Arsitektur Three-Tier/N-Tier}

Arsitektur three-tier (tiga lapisan) atau N-tier merupakan sebuah model arsitektur perangkat lunak yang membagi fungsionalitas aplikasi menjadi tiga lapisan logis dan fisik yang berbeda untuk meningkatkan skalabilitas dan keandalan sistem \cite{maruf2020,prabu2025}. Tiga lapisan utama tersebut terdiri dari: lapisan presentasi (\textit{presentation tier}), yang berinteraksi langsung dengan pengguna; lapisan aplikasi (\textit{application tier} atau \textit{business logic tier}), yang menangani pemorosesan data dan logika bisnis inti; dan lapisan data (\textit{data tier}), yang bertanggung jawab atas penyimpanan dan manajemen basis data \cite{maruf2020}. Pemisahan fungsionalitas ini memungkinkan setiap lapisan dikelola dan dikembangkan secara \textit{independent}, menjadi arsitektur ini pilihan yang efektif untuk sistem yang memerlukan ketersediaan tinggi (\textit{high availabilitiy}), seperti pada kasus penerapan LMS Moodle \cite{ismail2023}.

\subsection{Konsep Client-Server}

Client-server merupakan model perangkat lunak yang memungkinkan sumber daya dan permintaan layanan dipenuhi melalui jaringan, di mana komputer klien akan meminta layanan dan server akan menerima, memproses, serta memberikan respons yang sesuai \cite{assistantprofessor2020,geofrey2020}. Komunikasi antara klien dan server difasilitasi melalui protokol standar seperti HTTP, FTP, dan SMTP \cite{nyabuto2023}.

\section{REST API dan Protokol HTTP}

REST \textit{API} dipahami sebagai pendekatan arsitektur web service yang memanfaatkan prinsip \textit{Representational State Transfer} \cite{ehsan2022,roziqin2023}. REST menekankan penggunaan URI standar untuk mengidentifikasi resource, komunikasi stateless, serta penerapan uniform interface. HTTP berperan sebagai protokol utama yang digunakan sebagai standar komunikasi dengan metode GET, POST, PUT, dan DELETE \cite{aprilliyani2020,roziqin2023}.

\section{Autentikasi dan Otorisasi}

Bagian ini membahas konsep dasar autentikasi dan otorisasi yang menjadi fondasi penting dalam pengamanan aplikasi berbasis web. Mekanisme pengenalan identitas pengguna dan pemberian hak akses harus dirancang secara tepat agar interaksi antar sistem tetap aman, terukur, dan sesuai dengan tingkat kewenangan yang dibutuhkan. Oleh karena itu, pembahasan berikut mencakup OAuth 2.0 sebagai protokol delegasi akses, OpenID Connect sebagai lapisan identitas, JSON Web Token (JWT) sebagai format token yang umum digunakan, skema Bearer Token yang banyak diadopsi dalam komunikasi API, serta prinsip-prinsip keamanan API yang memastikan perlindungan terhadap ancaman dan penyalahgunaan akses.

\subsection{OAuth 2.0}

\textit{OAuth} 2.0 didefinisikan sebagai framework otorisasi yang memungkinkan suatu aplikasi memperoleh akses terbatas ke resource yang dilindungi tanpa harus menyimpan kredensial pengguna secara langsung \cite{lodderstedt2025}. OAuth 2.0 menyediakan seperangkat authorization server \cite{philippaerts2022}. Framework ini dirancang untuk mendukung berbagai konteks-mulai dari aplikasi web, single-page apps, hingga aplikasi mobile-dengan cara memberikan fleksibilitas pada mekanisme autentikasi dan otorisasi yang aman di antara berbagai jenis klien\cite{singh2023}.

\subsection{OpenID Connect}

OpenID Connect (OIDC) merupakan sebuah protokol yang mapan yang digunakan secara luas dalam manajemen identitas terfederasi (federated identity management). Protokol ini berfungsi sebagai dasar bagi otentikasi dan sistem Masuk Tunggal (Single Sign-On atau SSO), yang memungkinkan klien untuk memverifikasi identitas pengguna akhir berdasarkan otentikasi yang dilakukan oleh Penyedia Identitas (Identity Provider)\cite{hammann2020,yasuda2022}. Dibangun di atas kerangka kerja otorisasi OAuth 2.0, kegunaan OIDC meluas hingga ke infrastruktur kompleks, seperti memfasilitasi akses Secure Shell (SSH) pada pengaturan terfederasi dengan menggunakan token akses OIDC untuk otentikasi pengguna pada server jarak jauh\cite{gudu2025}.

\subsection{JWT}

\textit{JSON Web Token} (\textit{JWT}) merupakan sebuah standar terbuka yang didasarkan pada RFC 7519, yang digunakan secara luas sebagai mekanisme standar untuk otentikasi dan otorisasi pengguna pada layanan web. Standar ini tidak hanya populer untuk mengamankan transmisi data dan otentikasi pada RESTful API, tetapi juga dapat diperluas untuk meningkatkan keamanan dengan menyimpan informasi historis perilaku pengguna, seperti konsistensi alamat IP dan jenis user agent \cite{bucko2023,rahman2020}. Sementara itu, JWT secara fundamental adalah format token yang memfasilitasi transmisi data yang ringkas dan aman antara pihak-pihak yang berkepentingan sebagai objek JSON, yang menjadikannya mekanisme otentikasi yang penting dalam implementasi berbagai aplikasi modern \cite{nashikhuddin2023}.

\subsection{Skema Bearer Token}

Skema Bearer Token merupakan mekanisme autentikasi pada OAuth 2.0 di mana klien cukup menyertakan token pada header (Authorization: Bearer <token>) untuk memperoleh akses ke resource yang dilindungi \cite{lodderstedt2025}. Karena token ini bersifat bearer, siapa pun yang memilikinya dapat menggunakannya tanpa verifikasi tambahan, sehingga membuat keamanan transport menjadi aspek kritis. Penelitian terbaru menyoroti bahwa risiko pencurian token dapat diminimalkan melalui penggunaan kalal terenskripsi, pembatasan masa hidup token, serta validasi ketat pada sisi server \cite{ball2020}. Selain itu, praktik modern juga menekankan pentingnya menghindari pengiriman token melalui URL dan memastikan proses otorisasi mengikuti pedoman keamanan OAuth 2.0 \cite{neelan2022}.

\subsection{Keamanan API}

Keamanan API merupakan aspek kritis karena API sering menjadi target serangan \cite{chandramouli2020}. Banyak celah keamanan muncul akibat pengelolaan aset API yang lemah, API lama yang tidak terinventarisasi, serta kerentanan pada alur data dan logika bisnis \cite{sun2022}. Selain itu, meningkatnya kompleksitas arsitektur RESTful dan GraphQL memperluas permukaan serangan, termasuk risiko seperti information leakage, unauthorized access, dan eksploitasi validasi input yang tidak memadai \cite{zhao2020}

Untuk mengatasi ancaman tersebut, mekanisme keamanan API membutuhkan pendekatan berlapis yang mencakup autentikasi kuas berbasis OAuth/JWT, penggunaan HTTPS/TLS untuk mengamankan transmisi data, serta manjemen hak akses yang detail guna mencegah penyalahgunaan kredensial \cite{zhao2020}. Pentingnya teknik seperti asset discovery, traffic auditing, dan analisis alur data untuk mengidentifikasi API tersembunyi dan aktivitas mencurigakan \cite{sun2022}. Di samping itu, penggunaan API Gateaway dapat membantu menerapkan pembatasan trafik, filtrasi permintaan, dan perlindungan terhadap serangan seperti DDoS, sehingga API tetap terawasi dan terlindungi secara konsisten. 

\section{Database NoSQL (MongoDB)}

Basis data dokumen NoSQL (\textit{Not Only SQL}) muncul sebagai alternatif yang signifikan terhadap basis data relasional tradisional yang sering memiliki batasan ketat pada struktur data dan relasi, sehingga kurang efisien untuk menangani volume data yang sangat besar (\textit{huge database}) \cite{byali2022}. NoSQL document database mengatasi masalah ini dengan menyediakan kemampuan untuk menyimppan dan mengelola data dalam format dokumen, sehingga dapat menampung data yang tidak terstruktur, semi-struktur, maupun terstruktur \cite{carvalho2023}. Keunggulan utama NoSQL, khususnya jenis berorientasi dokumen seperti MongoDB, terletak pada fleksibilitas dan skalabilitas horizontal yang tinggi, menjadikannya pilihan esensial ketika skema data yang dinamis tidak sesuai dengan kebutuhan basis data realisonal \cite{byali2022}.

Secara opsional, basis data dokumen NoSQL menyimpan dalam bentuk dokumen. Meskipun memiliki perbedaan dalam beberapa aspek, MongoDB, Couchbase, dan CouchDB adalah contoh utama dari basis data dokume yang terkenal \cite{carvalho2023}. Sebagai contoh MongoDB merupakan basis data berorientasi dokumen, crossplatform, yang menawarkan kinjera tinggi, ketersediaan tinggi, dan skalabilitas yang sederhana. Basis data ini menggunakan MongoDB Query Language (MQL) yang dirancang untuk kemudahan penggunaan oleh pengembang \cite{byali2022}. Oleh karena itu, basis data dokumen NoSQL menjadi solusi penting untuk aplikasi padat data, memastikan penyimpanan big data dan kinerja kueri yang baik.

\section{Unified Modeling Language (UML)}

Unified Modeling Language (UML) didefinisikan sebagai sebuah bahasa pemodelan standar yang digunakan untuk merancang dan mendokumentasikan sistem berorientasi objek. Sebagai bahasa standar, UML menyediakan seperangkat notasi grafis yang komprehensif untuk memvisualisasikan, memspesifikasikan, membangun, dan mendokumentasikan artefak dalam sistem perangkat. Tujuan utama penggunaan UML adalah untuk memperjelas model yang tidak konsisten dan mengurangi ambiguitas selama proses pengembangan perangkat lunak \cite{amani2024}. UML membantu memvisualkan, menspesifikasikan, dan mendokumentasikan desain sistem secara grafis \cite{siska2024}.

Dengan menggunakan diagram-diagram yang berbeda, seperti Use Case Diagram dan Activity Diagram, UML membantu pengembang dalam memodelkan interaksi, struktur, dan perilaku sistem \cite{dabdawb2024}. Penerapan UML sangat krusial dalam siklus hidup pengembangan sistem (\textit{System Development Life Cycle} atau SDLC) karena membantu memastikan konsistensi model dan mempermudah komunikasi antara pihak-pihak yang terlibat dalam proyek \cite{marchezan2023}.

\subsection{Use Case Diagram}
Use Case adalah suatu diagram fundamental yang umum diajarkan dalam ilmu komputer dan rekayasa perangkat lunak. Diagram ini berfungsi sebagai representasi visual dari fungsionalitas sistem dari sudut pandang pengguna. Meskipun definisinya tampak sederhana, penilaian terhadap diagram use case sering kali menjadi hambatan dalam proses pembelajaran, terutama karena dua masalah utama: masalah interpersonal (tidak adanya standar penilaian di antara para pengajar) \cite{jebli2024} dan masalah intrapersonal (inkonsistensi seorang pengajar saat menilai banyak diagram) \cite{fauzan2021,abbott2025,wang2025}.

\subsection{Activity Diagram}
Activity Diagram adalah salah satu diagram perilaku yang tersedia dalam Unified Model Language (UML) yang digunakan untuk memodelkan alur kontrol dan alur data dalam suatu sistem \cite{sandfreni2021}. Diagram ini secara visual merepresentasikan Langkah-langkah, keputusan, dan urutan tindakan yang diperlukan untuk menyelesaikan suatu proses atau kegiatan bisnis tertentu \cite{siska2024}. Dalam konteks pemodelan sistem, Activity Diagram sangat berguna untuk memvisualisasikan bagaimana berbagai kegiatan saling terkait dan bergantung satu sama lain \cite{jha2023,ramdany2020}.

\subsection{Sequence Diagram}
Sequence Diagram adalah diagram UML yang paling umum kedua, digunakan untuk merepresentasikan interaksi objek dan pertukaran pesan antar objek tersebut seiring berjalannya waktu \cite{siska2024}. Diagram ini secara visual menunjukkan bagaimana peristiwa atau aktivitas yang ada dalam sebuah use case dipetakan menjadi operasi-operasi dari kelas objek yang ada pada Class Diagram \cite{alfedaghi2020}.

\subsection{Class Diagram}
Class Diagram merupakan salah satu diagram Unified Modeling Language (UML) yang paling umum digunakan dalam Pendidikan dan pengembangan perangkat lunak berorientasi objek \cite{siska2024}. Fungsi utama dari Class Diagram adalah untuk merepresentasikan kelas-kelas dalam sistem perangkat lunak dan hubungan yang terjalin antar kelas-kelas tersebut \cite{fauzan2021}.


\section{Analisis Sistem pada Aplikasi}

Dalam pengembangan aplikasi, analisis terhadap sistem yang berjalan diperlukan untuk memahami alur kerja secara menyeluruh sebagai dasar usulan pengembangan. Analisis dilakukan dengan merujuk pada alur bisnis atau urutan operasional sistem yang digambarkan melalui use case diagram, activity diagram, sequence diagram, dan deployment diagram. Pemahaman yang jelas terhadap alur bisnis akan mendukung pengembangan sistem yang lebih efektif.

\subsection{Analisis Sistem yang Sedang Berjalan}

Sistem pada aplikasi MedisLink dimulai ketika pengguna membutuhkan alat bantu medis. Pengguna melakukan registrasi dan login, kemudian memilih alat berdasarkan kategori yang tersedia. Sistem menampilkan informasi ketersediaan alat untuk menentukan apakah alat dapat dipinjam.

Jika alat tidak tersedia, pengguna dapat menunggu atau memilih alternatif lain. Jika tersedia, pengguna mengisi data peminjaman sesuai ketentuan. Setelah peminjaman berhasil, alat digunakan selama periode tertentu. Sistem mengirimkan notifikasi pengingat sebelum batas waktu pengembalian.

Apabila alat dikembalikan tepat waktu, proses dinyatakan selesai dan data dicatat. Jika terjadi keterlambatan, status diperbarui hingga alat dikembalikan. Proses berakhir ketika alat siap dipinjam kembali oleh pengguna lain.

\section{Analisis Sistem}

\subsection{Analisis Sistem yang Akan Dibangun}

Sistem dimulai dengan registrasi dan login pengguna. Setelah berhasil masuk, pengguna diarahkan ke dashboard untuk melihat informasi akun dan daftar alat bantu medis. Pengguna dapat memilih alat berdasarkan kategori, melihat detail, dan mengajukan peminjaman secara online.

Sistem memproses permintaan dengan memeriksa ketersediaan dan mencatat transaksi. Selama masa peminjaman, sistem mengirim notifikasi pengingat sesuai jadwal. Pengguna dapat melihat status serta riwayat peminjaman.

Admin bertugas mengelola data pengguna, alat, kategori, serta memantau proses peminjaman dan pengembalian. Proses berakhir saat alat dikembalikan dan status dinyatakan selesai.

\subsection{Kebutuhan Fungsional}

Berikut kebutuhan fungsional sistem:

\begin{enumerate}
    \item Aplikasi berbasis website yang dapat diakses melalui internet.
    \item Fitur pendaftaran dan login pengguna.
    \item Menampilkan daftar alat bantu medis berdasarkan kategori.
    \item Menampilkan detail alat sebelum peminjaman.
    \item Pengajuan peminjaman secara online.
    \item Pencatatan dan penyimpanan transaksi peminjaman.
    \item Notifikasi pengingat pengembalian alat.
    \item Riwayat peminjaman pengguna.
    \item Fitur ulasan dan penilaian setelah peminjaman.
    \item Admin dapat mengelola alat, kategori, serta memantau peminjaman dan pengembalian.
\end{enumerate}

\subsection{Kebutuhan Nonfungsional}

\textbf{Perangkat Lunak:}
\begin{itemize}
    \item Sistem Operasi: Windows 11 atau Linux Ubuntu 20.04
    \item Editor Kode: Visual Studio Code
    \item Browser: Google Chrome atau Mozilla Firefox
    \item Database: MongoDB
    \item Backend: Golang dengan framework Fiber
    \item Frontend: React.js
    \item Autentikasi: JSON Web Token (JWT)
\end{itemize}

\textbf{Perangkat Keras:}
\begin{itemize}
    \item Prosesor: Intel Core i5 Gen-10 atau AMD Ryzen 5 5000 series
    \item RAM: Minimal 8 GB
    \item Penyimpanan: SSD 256 GB
    \item Koneksi internet stabil
\end{itemize}

\section{Use Case}

\begin{figure}[H]
    \centering
    \includegraphics[width=\textwidth]{images/bab3/usecase.png}
    \caption{Use Case Diagram}
    \label{fig:usecase_bab3}
\end{figure}

Gambar~\ref{fig:usecase_bab3} menggambarkan interaksi antara aktor dan sistem MedisLink dalam menjalankan seluruh fitur aplikasi. Terdapat tiga aktor utama, yaitu Tamu, Pengguna, dan Admin, yang masing-masing memiliki peran berbeda dalam sistem.

Tamu hanya dapat melihat halaman \textit{landing page} tanpa perlu login. Pengguna memiliki akses untuk melakukan register, login, melihat riwayat donasi, mendonasikan alat, melakukan \textit{forgot password}, melihat notifikasi, melihat riwayat peminjaman, meminjam alat, serta mengelola akun.

Admin bertanggung jawab mengelola inventaris, melakukan konfirmasi peminjaman dan donasi, serta mengelola berita dan iklan.

Proses \textit{Login} menjadi use case utama karena hampir seluruh fitur lainnya memiliki relasi \textit{include} terhadap login, yang berarti autentikasi diperlukan sebelum fitur dapat diakses. Diagram ini menunjukkan sistem dirancang dengan kontrol akses yang terstruktur sesuai peran masing-masing aktor.

\section{Class Diagram}

\begin{figure}[H]
    \centering
    \includegraphics[width=\textwidth]{images/bab3/classdiagram.png}
    \caption{Class Diagram}
    \label{fig:class_bab3}
\end{figure}

Gambar~\ref{fig:class_bab3} menunjukkan struktur class utama beserta hubungan antar class dalam sistem MedisLink.

Class \textit{User} berperan sebagai entitas utama yang dapat melakukan registrasi, login, melihat alat, mengajukan peminjaman, melakukan donasi, serta menerima notifikasi. Class \textit{Loan} merepresentasikan proses peminjaman alat medis yang menghubungkan User dengan class \textit{Tool}, yang menyimpan data alat seperti kategori, kondisi, stok, dan status.

Class \textit{Donation} mencatat proses donasi alat medis yang dilakukan oleh user, termasuk informasi alat dan status donasi. Class \textit{Admin} bertanggung jawab dalam pengelolaan sistem seperti mengelola inventaris, menyetujui atau menolak peminjaman dan donasi, serta mengelola konten berita.

Class \textit{Notification} digunakan untuk mengirim informasi terkait status peminjaman dan donasi kepada user, sedangkan class \textit{News} berfungsi untuk mengelola dan menampilkan informasi pada aplikasi.

Relasi antar class menunjukkan bahwa satu user dapat memiliki banyak peminjaman, donasi, dan notifikasi, sementara admin berperan dalam validasi dan pengelolaan data sistem.

\section{Activity Diagram}

Activity diagram digunakan untuk memodelkan perilaku sistem dalam suatu alur proses. Diagram ini menggambarkan interaksi elemen dinamis serta jalur logis berdasarkan kondisi, percabangan, maupun proses paralel.

\subsection{Tamu Melihat Landing Page}

\begin{figure}[H]
    \centering
    \includegraphics[width=0.9\textwidth]{images/bab3/activity_tamu.png}
    \caption{Activity Diagram Tamu Melihat Landing Page}
    \label{fig:activity_tamu}
\end{figure}

Gambar~\ref{fig:activity_tamu} menunjukkan alur interaksi antara aktor Tamu dan sistem MedisLink. Proses dimulai ketika tamu membuka aplikasi tanpa melakukan login. Sistem kemudian menampilkan halaman \textit{landing page} sebagai halaman awal aplikasi. Proses berakhir setelah halaman berhasil ditampilkan tanpa akses ke fitur internal sistem.
\subsection{Registrasi}

\begin{figure}[H]
    \centering
    \includegraphics[width=0.9\textwidth]{images/bab3/activity_registrasi.png}
    \caption{Activity Diagram Registrasi}
    \label{fig:activity_registrasi}
\end{figure}

Gambar~\ref{fig:activity_registrasi} menggambarkan alur proses pendaftaran pengguna dalam sistem. Proses dimulai ketika pengguna membuka halaman registrasi yang ditampilkan oleh sistem, kemudian pengguna mengisi data yang diperlukan.

Setelah data diinput, sistem melakukan validasi untuk memastikan kelengkapan dan kebenaran data. Jika data tidak valid, sistem menampilkan pesan kesalahan. Jika data valid, sistem mengarahkan pengguna ke halaman login. Diagram ini menunjukkan interaksi antara pengguna dan sistem berdasarkan kondisi yang terjadi.

\subsection{Login User}

\begin{figure}[H]
    \centering
    \includegraphics[width=0.9\textwidth]{images/bab3/activity_login.png}
    \caption{Activity Diagram Login User}
    \label{fig:activity_login}
\end{figure}

Gambar~\ref{fig:activity_login} menggambarkan proses autentikasi pengguna dalam sistem. Proses dimulai ketika sistem menampilkan form login dan pengguna menginput data yang diperlukan.

Sistem kemudian melakukan validasi terhadap data login. Jika data valid, sistem menampilkan halaman dashboard. Jika data tidak valid, sistem menampilkan pesan kesalahan dan mengarahkan pengguna kembali ke halaman login. Diagram ini menunjukkan interaksi antara pengguna dan sistem berdasarkan hasil validasi.

\subsection{Peminjaman Alat Bantu Medis}

\begin{figure}[H]
    \centering
    \includegraphics[width=0.9\textwidth]{images/bab3/activity_peminjaman.png}
    \caption{Activity Diagram Peminjaman Alat Bantu Medis}
    \label{fig:activity_peminjaman}
\end{figure}

Gambar~\ref{fig:activity_peminjaman} menggambarkan alur proses pengajuan peminjaman alat oleh pengguna dalam sistem. Proses dimulai dari dashboard, di mana pengguna memilih opsi untuk menjelajahi inventaris dan memilih alat yang diinginkan.

Sistem kemudian menampilkan form input peminjaman. Setelah pengguna mengisi data, sistem melakukan validasi untuk memastikan kesesuaian data. Jika data tidak sesuai, pengguna diminta untuk mengisi kembali. Jika data valid, sistem mengirimkan pengajuan kepada admin sebagai tahap akhir proses peminjaman.
\subsection{Donasi Alat}

\begin{figure}[H]
    \centering
    \includegraphics[width=0.9\textwidth]{images/bab3/activity_donasi.png}
    \caption{Activity Diagram Donasi Alat}
    \label{fig:activity_donasi}
\end{figure}

Gambar~\ref{fig:activity_donasi} menggambarkan proses donasi alat medis dalam sistem MedisLink. Proses dimulai ketika admin mengakses dashboard dan memilih menu donasi alat.

Sistem kemudian menampilkan form input donasi yang harus diisi. Setelah data diinput, sistem melakukan validasi. Jika data belum sesuai, admin diminta memperbaiki input. Jika data valid, sistem memproses data dan status donasi menunggu konfirmasi hingga proses selesai.
\subsection{Riwayat Peminjaman User}

\begin{figure}[H]
    \centering
    \includegraphics[width=0.9\textwidth]{images/bab3/activity_riwayat_peminjaman.png}
    \caption{Activity Diagram Riwayat Peminjaman}
    \label{fig:activity_riwayat_peminjaman}
\end{figure}

Gambar~\ref{fig:activity_riwayat_peminjaman} menggambarkan proses melihat riwayat transaksi peminjaman. Pengguna memulai dari dashboard dan memilih menu riwayat pinjaman. Sistem kemudian menampilkan riwayat peminjaman yang telah dilakukan. Proses berakhir setelah data berhasil ditampilkan.


\subsection{Riwayat Donasi User}

\begin{figure}[H]
    \centering
    \includegraphics[width=0.9\textwidth]{images/bab3/activity_riwayat_donasi.png}
    \caption{Activity Diagram Riwayat Donasi User}
    \label{fig:activity_riwayat_donasi}
\end{figure}

Gambar~\ref{fig:activity_riwayat_donasi} menggambarkan alur aktivitas pengguna dalam melihat riwayat donasi. Pengguna memilih menu riwayat donasi pada dashboard, kemudian sistem menampilkan daftar riwayat donasi yang pernah dilakukan. Proses berakhir setelah riwayat berhasil ditampilkan.
\subsection{User Melihat Notifikasi}

\begin{figure}[H]
    \centering
    \includegraphics[width=0.9\textwidth]{images/bab3/activity_notifikasi.png}
    \caption{Activity Diagram User Melihat Notifikasi}
    \label{fig:activity_notifikasi}
\end{figure}

Gambar~\ref{fig:activity_notifikasi} menunjukkan proses ketika user berada di dashboard dan memilih menu notifikasi. Sistem kemudian menampilkan daftar notifikasi terkait aktivitas peminjaman atau donasi. Proses berakhir setelah notifikasi berhasil ditampilkan.


\subsection{Mengelola Akun}

\begin{figure}[H]
    \centering
    \includegraphics[width=0.9\textwidth]{images/bab3/activity_kelola_akun.png}
    \caption{Activity Diagram Mengelola Akun}
    \label{fig:activity_kelola_akun}
\end{figure}

Gambar~\ref{fig:activity_kelola_akun} menggambarkan proses pembaruan data pribadi pengguna. Pengguna memilih ikon akun, sistem menampilkan form detail data, kemudian pengguna memperbarui data. Sistem melakukan validasi; jika data tidak valid, pengguna diminta memperbaiki input. Jika valid, sistem menampilkan pesan sukses sebagai tanda pembaruan berhasil.
\subsection{Login Admin}

\begin{figure}[H]
    \centering
    \includegraphics[width=0.9\textwidth]{images/bab3/activity_login_admin.png}
    \caption{Activity Diagram Login Admin}
    \label{fig:activity_login_admin}
\end{figure}

Gambar~\ref{fig:activity_login_admin} menggambarkan proses login admin dalam aplikasi MedisLink. Admin menginput data login, kemudian sistem melakukan validasi. Jika data valid, sistem menampilkan halaman dashboard admin. Jika tidak valid, sistem menampilkan pesan kesalahan dan mengarahkan kembali ke halaman login.


\subsection{Admin Mengelola Peminjaman}

\begin{figure}[H]
    \centering
    \includegraphics[width=0.95\textwidth]{images/bab3/activity_kelola_peminjaman_admin.png}
    \caption{Activity Diagram Admin Mengelola Peminjaman}
    \label{fig:activity_kelola_peminjaman_admin}
\end{figure}

Gambar~\ref{fig:activity_kelola_peminjaman_admin} menggambarkan proses pengelolaan peminjaman oleh admin. Admin memilih menu manajemen peminjaman, kemudian sistem menampilkan daftar peminjaman untuk divalidasi.

Admin dapat menyetujui atau menolak peminjaman, dan sistem mengirim notifikasi kepada pengguna sesuai keputusan tersebut. Jika disetujui, admin mengonfirmasi penyerahan barang dan pengembalian barang hingga sistem menampilkan status bahwa peminjaman telah selesai.
\subsection{Admin Mengelola Inventaris}

\begin{figure}[H]
    \centering
    \includegraphics[width=0.9\textwidth]{images/bab3/activity_kelola_inventaris.png}
    \caption{Activity Diagram Admin Mengelola Inventaris}
    \label{fig:activity_kelola_inventaris}
\end{figure}

Gambar~\ref{fig:activity_kelola_inventaris} menggambarkan proses penambahan data alat medis oleh admin. Admin memilih menu inventaris dan menekan tombol tambah alat, kemudian sistem menampilkan form input. Setelah data dimasukkan, sistem melakukan validasi. Jika data tidak sesuai, admin diminta memperbaiki input. Jika valid, sistem menyimpan data dan menampilkan notifikasi bahwa penambahan alat berhasil.


\subsection{Admin Mengelola Berita}

\begin{figure}[H]
    \centering
    \includegraphics[width=0.9\textwidth]{images/bab3/activity_kelola_berita.png}
    \caption{Activity Diagram Admin Mengelola Berita}
    \label{fig:activity_kelola_berita}
\end{figure}

Gambar~\ref{fig:activity_kelola_berita} menggambarkan proses pengelolaan berita oleh admin. Admin memilih menu manajemen berita, kemudian sistem menampilkan halaman pengelolaan. Admin mengisi data berita dan sistem menampilkan berita kepada pengguna setelah proses selesai.
\subsection{Admin Mengelola Iklan}

\begin{figure}[H]
    \centering
    \includegraphics[width=0.9\textwidth]{images/bab3/activity_kelola_iklan.png}
    \caption{Activity Diagram Admin Mengelola Iklan}
    \label{fig:activity_kelola_iklan}
\end{figure}

Gambar~\ref{fig:activity_kelola_iklan} menggambarkan proses pengelolaan iklan oleh admin. Admin memilih menu manajemen iklan pada dashboard, kemudian sistem menampilkan halaman pengelolaan iklan. Admin mengisi data iklan yang akan dipublikasikan dan sistem menampilkan iklan setelah proses selesai.


\subsection{Admin Konfirmasi Donasi}

\begin{figure}[H]
    \centering
    \includegraphics[width=0.95\textwidth]{images/bab3/activity_konfirmasi_donasi.png}
    \caption{Activity Diagram Admin Konfirmasi Donasi}
    \label{fig:activity_konfirmasi_donasi}
\end{figure}

Gambar~\ref{fig:activity_konfirmasi_donasi} menggambarkan proses admin dalam mengonfirmasi donasi alat medis. Admin memilih menu donasi, sistem menampilkan daftar donasi, kemudian admin dapat melihat detail barang sebelum melakukan konfirmasi. Setelah donasi dikonfirmasi, sistem menampilkan notifikasi bahwa donasi berhasil dan proses dinyatakan selesai.
\section{Sequence Diagram}

Sequence Diagram digunakan untuk menggambarkan interaksi antar objek dalam sistem berdasarkan urutan waktu eksekusi. Diagram ini menunjukkan komunikasi antar objek pada setiap use case yang dijalankan.

\subsection{Tamu Melihat Landing Page}

\begin{figure}[H]
    \centering
    \includegraphics[width=\textwidth]{images/bab3/sequence_tamu.png}
    \caption{Sequence Diagram Tamu Melihat Landing Page}
    \label{fig:sequence_tamu}
\end{figure}

Gambar~\ref{fig:sequence_tamu} menggambarkan interaksi antara Tamu, Aplikasi, dan Sistem saat mengakses MedisLink. Proses dimulai ketika tamu membuka aplikasi, kemudian aplikasi meminta sistem untuk menampilkan halaman awal. Sistem merespons dengan mengirimkan halaman landing page yang kemudian ditampilkan kepada tamu. Proses berakhir setelah landing page berhasil ditampilkan.
\subsection{Login}

\begin{figure}[H]
    \centering
    \includegraphics[width=\textwidth]{images/bab3/sequence_login.png}
    \caption{Sequence Diagram Login}
    \label{fig:sequence_login}
\end{figure}

Gambar~\ref{fig:sequence_login} menggambarkan proses autentikasi pengguna atau admin. Aktor memasukkan username dan password melalui aplikasi, kemudian sistem memverifikasi data ke database. Jika data tidak valid, sistem mengembalikan pesan kesalahan. Jika valid, sistem menampilkan halaman utama sesuai peran pengguna.


\subsection{Peminjaman Alat Bantu Medis}

\begin{figure}[H]
    \centering
    \includegraphics[width=\textwidth]{images/bab3/sequence_peminjaman.png}
    \caption{Sequence Diagram Peminjaman Alat Bantu Medis}
    \label{fig:sequence_peminjaman}
\end{figure}

Gambar~\ref{fig:sequence_peminjaman} menggambarkan proses pengajuan peminjaman. User menginput data melalui aplikasi, kemudian sistem melakukan validasi. Jika gagal, sistem menampilkan pesan kesalahan. Jika berhasil, sistem menyimpan data ke database dan menampilkan pesan sukses kepada user.


\subsection{User Donasi Alat}

\begin{figure}[H]
    \centering
    \includegraphics[width=\textwidth]{images/bab3/sequence_donasi.png}
    \caption{Sequence Diagram Donasi Alat}
    \label{fig:sequence_donasi}
\end{figure}

Gambar~\ref{fig:sequence_donasi} menggambarkan proses donasi alat oleh user. Data donasi dikirim ke sistem untuk divalidasi. Jika tidak valid, sistem mengembalikan pesan kesalahan. Jika valid, data disimpan ke database dan sistem menampilkan pesan sukses.
\subsection{Riwayat Peminjaman User}

\begin{figure}[H]
    \centering
    \includegraphics[width=\textwidth]{images/bab3/sequence_riwayat_peminjaman.png}
    \caption{Sequence Diagram Riwayat Peminjaman User}
    \label{fig:sequence_riwayat_peminjaman}
\end{figure}

Gambar~\ref{fig:sequence_riwayat_peminjaman} menggambarkan proses user dalam mengakses riwayat peminjaman. Aplikasi meminta data ke sistem, sistem mengambil data dari database, kemudian data dikirim kembali ke aplikasi untuk ditampilkan kepada user.


\subsection{Riwayat Donasi User}

\begin{figure}[H]
    \centering
    \includegraphics[width=\textwidth]{images/bab3/sequence_riwayat_donasi.png}
    \caption{Sequence Diagram Riwayat Donasi User}
    \label{fig:sequence_riwayat_donasi}
\end{figure}

Gambar~\ref{fig:sequence_riwayat_donasi} menggambarkan proses user dalam melihat riwayat donasi. Sistem mengambil data dari database dan meneruskannya ke aplikasi untuk ditampilkan kepada user.


\subsection{Mengelola Akun User}

\begin{figure}[H]
    \centering
    \includegraphics[width=\textwidth]{images/bab3/sequence_kelola_akun.png}
    \caption{Sequence Diagram Mengelola Akun User}
    \label{fig:sequence_kelola_akun}
\end{figure}

Gambar~\ref{fig:sequence_kelola_akun} menggambarkan proses pembaruan data pengguna. Data dikirim ke sistem untuk divalidasi. Jika gagal, sistem menampilkan pesan kesalahan. Jika berhasil, data disimpan ke database dan sistem menampilkan pesan sukses.
\subsection{User Melihat Notifikasi}

\begin{figure}[H]
    \centering
    \includegraphics[width=\textwidth]{images/bab3/sequence_notifikasi.png}
    \caption{Sequence Diagram User Melihat Notifikasi}
    \label{fig:sequence_notifikasi}
\end{figure}

Gambar~\ref{fig:sequence_notifikasi} menggambarkan proses user dalam melihat notifikasi. Aplikasi meminta data notifikasi ke sistem, sistem mengambil data dari database, kemudian data dikirim kembali ke aplikasi untuk ditampilkan kepada user.


\subsection{Admin Mengelola Peminjaman}

\begin{figure}[H]
    \centering
    \includegraphics[width=\textwidth]{images/bab3/sequence_admin_peminjaman.png}
    \caption{Sequence Diagram Admin Mengelola Peminjaman}
    \label{fig:sequence_admin_peminjaman}
\end{figure}

Gambar~\ref{fig:sequence_admin_peminjaman} menunjukkan proses validasi peminjaman oleh admin. Sistem mengambil daftar peminjaman dari database, admin menyetujui atau menolak, kemudian sistem memperbarui status dan mengirim notifikasi. Proses berlanjut hingga konfirmasi pengembalian dan status berubah menjadi selesai serta stok alat diperbarui.


\subsection{Admin Mengelola Inventaris}

\begin{figure}[H]
    \centering
    \includegraphics[width=\textwidth]{images/bab3/sequence_admin_inventaris.png}
    \caption{Sequence Diagram Admin Mengelola Inventaris}
    \label{fig:sequence_admin_inventaris}
\end{figure}

Gambar~\ref{fig:sequence_admin_inventaris} menggambarkan proses penambahan alat oleh admin. Sistem menampilkan daftar alat, admin mengisi form tambah alat, kemudian sistem memvalidasi dan menyimpan data ke database. Setelah berhasil, sistem menampilkan notifikasi keberhasilan.
\subsection{Admin Mengelola Berita}

\begin{figure}[H]
    \centering
    \includegraphics[width=\textwidth]{images/bab3/sequence_admin_berita.png}
    \caption{Sequence Diagram Admin Mengelola Berita}
    \label{fig:sequence_admin_berita}
\end{figure}

Gambar~\ref{fig:sequence_admin_berita} menunjukkan proses publikasi berita oleh admin. Sistem menampilkan daftar berita dari database, kemudian admin mengisi dan menyimpan berita baru. Setelah database mengonfirmasi penyimpanan berhasil, sistem menampilkan berita tersebut pada halaman manajemen.


\subsection{Admin Mengelola Iklan}

\begin{figure}[H]
    \centering
    \includegraphics[width=\textwidth]{images/bab3/sequence_admin_iklan.png}
    \caption{Sequence Diagram Admin Mengelola Iklan}
    \label{fig:sequence_admin_iklan}
\end{figure}

Gambar~\ref{fig:sequence_admin_iklan} menggambarkan proses publikasi iklan oleh admin. Sistem menampilkan data iklan dari database, admin mengisi dan menyimpan data baru, kemudian sistem memperbarui tampilan setelah penyimpanan berhasil.


\subsection{Admin Konfirmasi Donasi}

\begin{figure}[H]
    \centering
    \includegraphics[width=\textwidth]{images/bab3/sequence_admin_konfirmasi_donasi.png}
    \caption{Sequence Diagram Admin Konfirmasi Donasi}
    \label{fig:sequence_admin_konfirmasi_donasi}
\end{figure}

Gambar~\ref{fig:sequence_admin_konfirmasi_donasi} menunjukkan proses validasi donasi oleh admin. Sistem menampilkan daftar donasi pending dari database, admin meninjau detail dan melakukan konfirmasi. Sistem kemudian memperbarui status donasi menjadi disetujui serta menambahkan stok barang ke inventaris. Proses berakhir ketika notifikasi keberhasilan ditampilkan.
\section{Deployment}

\begin{figure}[H]
    \centering
    \includegraphics[width=\textwidth]{images/bab3/deployment.png}
    \caption{Deployment}
    \label{fig:deployment}
\end{figure}

Gambar~\ref{fig:deployment} menunjukkan arsitektur infrastruktur sistem MedisLink. Pengguna mengakses aplikasi melalui web browser pada perangkat PC atau HP. Aset frontend (React SPA) dihosting pada Vercel dan diakses melalui protokol HTTPS. Permintaan data dikirim melalui REST API ke backend MedisLink API yang berjalan pada Railway atau Render menggunakan protokol HTTPS. Backend terhubung secara aman melalui koneksi TCP/TLS ke MongoDB Atlas sebagai basis data utama. Selain itu, sistem juga terintegrasi dengan Cloudinary untuk penyimpanan dan pengelolaan aset gambar. Arsitektur ini memastikan keamanan komunikasi data serta pemisahan layanan frontend, backend, dan storage.

\chapter{IMPLEMENTASI DAN PENGUJIAN}

\section{Pembahasan Hasil Implementasi}

\subsection{Landing Page}

\begin{figure}[H]
    \centering
    \includegraphics[width=\textwidth]{images/bab4/landing_page.png}
    \caption{Halaman Landing Page}
    \label{fig:landing_page}
\end{figure}

Gambar~\ref{fig:landing_page} merupakan halaman utama aplikasi MedisLink yang pertama kali ditampilkan kepada pengguna. Halaman ini berfungsi sebagai pengantar aplikasi dan menampilkan informasi singkat mengenai layanan peminjaman alat bantu medis. Pada bagian atas terdapat tombol \textit{Masuk/Daftar} yang digunakan untuk proses autentikasi pengguna. Selain itu, tersedia tombol \textit{Jelajahi Alat} untuk melihat daftar alat bantu medis serta tombol \textit{Hubungi Kami} untuk memperoleh informasi lebih lanjut mengenai layanan yang tersedia.

\chapter{Kesimpulan}

Berdasarkan hasil perancangan, implementasi, dan pengujian yang telah dilakukan, dapat disimpulkan bahwa platform Nyumbangin berhasil dikembangkan sebagai prototipe platform donasi digital berbasis web yang fungsional. Sistem mampu menangani alur utama donasi, mulai dari autentikasi pengguna, pencatatan transaksi, integrasi pembayaran, notifikasi real-time melalui overlay, hingga mekanisme pencairan dana (payout) bagi kreator secara terstruktur.

Fitur-fitur inti yang dirancang pada tahap analisis telah diimplementasikan dan berjalan sesuai kebutuhan. Penggunaan arsitektur aplikasi web modern serta pemodelan sistem menggunakan UML dan ERD membantu memastikan alur proses dan struktur data berjalan konsisten. Hasil pengujian menunjukkan bahwa proses kritis seperti donasi, pembayaran, webhook, dan payout dapat berjalan stabil dan menghasilkan data yang valid.

Pengujian sistem difokuskan pada modul-modul utama yang bersifat kritis, sehingga hasil \textit{code coverage} yang diperoleh hanya merepresentasikan pengujian fitur inti dan tidak mencakup seluruh modul aplikasi, khususnya antarmuka pengguna dan utilitas pendukung. Meskipun demikian, sistem telah memenuhi tujuan pengembangan sebagai platform donasi digital yang operasional. Pengembangan lanjutan masih dapat dilakukan, terutama pada perluasan cakupan pengujian, peningkatan keamanan, dan optimasi performa untuk penggunaan pada skala yang lebih besar.


\include{chapters/literature_review}
\include{chapters/methodology}
\include{chapters/results}
\include{chapters/conclusion}

% --- BIBLIOGRAPHY ---
\nocite{*}
\cleardoublepage
\phantomsection
\addcontentsline{toc}{chapter}{\bibname}
\bibliographystyle{plain}
\bibliography{references}

\end{document}
