\documentclass[12pt,a4paper]{report}

% ================== PACKAGE ==================
\usepackage[utf8]{inputenc}
\usepackage[indonesian]{babel}
\usepackage{graphicx}
\usepackage{setspace}
\usepackage{geometry}
\usepackage{titlesec}
\usepackage{tocloft}
\usepackage{hyperref}
\usepackage{apacite}
\usepackage{caption}

\geometry{left=4cm, right=3cm, top=3cm, bottom=3cm}
\onehalfspacing

% ================== BEGIN DOCUMENT ==================
\begin{document}

% ================== COVER ==================
\begin{titlepage}
\centering
\vspace{2cm}

{\Large \textbf{SISTEM PEMINJAMAN ALAT BANTU MEDIS BERBASIS WEB}}\\
{\Large \textbf{MENGGUNAKAN REACT JS DAN GO}}

\vspace{2cm}

Diajukan untuk Memenuhi Kelulusan Matakuliah Proyek 3\\
Program Studi DIV Teknik Informatika

\vspace{2cm}

\textbf{DISUSUN OLEH :}\\
Moch Restu Agis Burhanudin (714230059)\\
Aghni Hasna Mufida (714230069)

\vfill

PROGRAM STUDI DIV TEKNIK INFORMATIKA\\
UNIVERSITAS LOGISTIK DAN BISNIS INTERNASIONAL\\
BANDUNG\\
2026

\end{titlepage}

% ================== LEMBAR PENGESAHAN ==================
\chapter*{LEMBAR PENGESAHAN}
\addcontentsline{toc}{chapter}{LEMBAR PENGESAHAN}

Laporan Proyek III ini telah diperiksa, disetujui dan disidangkan di Bandung, .... / .... / 2026.

\vspace{1cm}

Penguji Pendamping \hfill Penguji Utama\\
Rolly Maulana Awangga, S.T., M.T \hfill Nisa Hanum Harani, S.Kom., M.T.

\vspace{1cm}

Pembimbing \hfill Koordinator Proyek II\\
Nisa Hanum Harani, S.Kom., M.T. \hfill Roni Habibi, S.Kom., M.T., SFPC

\vspace{1cm}

Mengetahui,\\
Ketua Program Studi D-IV Teknik Informatika\\
Roni Andarsyah, S.T., M.Kom.

\clearpage

% ================== ABSTRAK ==================
\chapter*{ABSTRAK}
\addcontentsline{toc}{chapter}{ABSTRAK}

Sebagian besar masyarakat menghadapi masalah aksesibilitas alat bantu medis...
(isi lengkap abstrak Anda di sini)

\textbf{Kata Kunci:} MedisLink, React.js, Go (Fiber), MongoDB, JWT, Cron Job.

\clearpage

% ================== ABSTRACT ==================
\chapter*{ABSTRACT}
\addcontentsline{toc}{chapter}{ABSTRACT}

Most people face accessibility issues for medical devices...
(isi lengkap abstract Anda di sini)

\textbf{Keywords:} MedisLink, React.js, Go (Fiber), MongoDB, JWT, Cron Job.

\clearpage

% ================== KATA PENGANTAR ==================
\chapter*{KATA PENGANTAR}
\addcontentsline{toc}{chapter}{KATA PENGANTAR}

Puji syukur kami panjatkan kehadirat Tuhan Yang Maha Esa...
(isi lengkap kata pengantar)

\clearpage

% ================== DAFTAR ISI ==================
\tableofcontents
\listoftables
\listoffigures
\clearpage

% ================== BAB I ==================
\chapter{PENDAHULUAN}

\section{Latar Belakang}
Keterbatasan akses terhadap alat bantu medis di Indonesia masih menjadi masalah...

\section{Deskripsi Aplikasi}
Aplikasi MedisLink adalah platform berbasis web...

\section{Identifikasi Masalah}
\begin{enumerate}
\item Bagaimana membangun sistem peminjaman alat bantu medis...
\item Bagaimana sistem ini membantu keterbatasan stok...
\item Bagaimana memfasilitasi donasi alat medis...
\end{enumerate}

\section{Tujuan}
\begin{enumerate}
\item Merancang dan membangun platform peminjaman alat bantu medis...
\item Membangun sistem manajemen inventaris...
\item Menyediakan fitur donasi...
\end{enumerate}

\section{Ruang Lingkup}
Pengembangan sistem menggunakan Golang (Fiber), React.js, MongoDB...

% ================== BAB II ==================
\chapter{TINJAUAN PUSTAKA}

\section{Landasan Teori}
\subsection{API}
REST API adalah kerangka kerja...

\subsection{Node.js}
Node.js adalah runtime JavaScript...

\subsection{Go}
Go adalah bahasa pemrograman open-source...

\subsection{React JS}
React.js adalah pustaka JavaScript...

\subsection{MongoDB}
MongoDB adalah database NoSQL berbasis dokumen...

\subsection{UML}
Unified Modeling Language (UML) merupakan bahasa pemodelan...

% ================== BAB III ==================
\chapter{ANALISIS DAN PERANCANGAN}

\section{Analisis Sistem}
Sistem berjalan dimulai ketika pengguna melakukan registrasi...

\section{Kebutuhan Fungsional}
\begin{itemize}
\item Sistem berbasis web
\item Fitur login dan registrasi
\item Peminjaman alat
\item Notifikasi pengembalian
\item Manajemen inventaris
\end{itemize}

\section{Kebutuhan Nonfungsional}
\begin{itemize}
\item OS: Windows 11 / Ubuntu
\item Backend: Golang Fiber
\item Frontend: React.js
\item Database: MongoDB
\end{itemize}

% ================== BAB IV ==================
\chapter{IMPLEMENTASI DAN PENGUJIAN}

\section{Implementasi}
Landing page, login, register, dashboard, peminjaman alat...

\section{Pengujian}
Pengujian dilakukan dengan metode blackbox testing...

% ================== BAB V ==================
\chapter{KESIMPULAN DAN SARAN}

\section{Kesimpulan}
Sistem MedisLink mampu menghubungkan pasien dan donatur...

\section{Saran}
Pengembangan selanjutnya dapat berupa integrasi WhatsApp Gateway...

% ================== DAFTAR PUSTAKA ==================
\bibliographystyle{apacite}
\bibliography{referensi}

\end{document}
