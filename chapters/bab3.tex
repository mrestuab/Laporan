\chapter{Metode Penelitian}

\section{Metode Pengembangan Sistem}

Pengembangan platform Nyumbangin menggunakan pendekatan Agile, yang menekankan proses pengembangan sistem secara bertahap, adaptif, dan berulang. Pendekatan Agile dipilih karena sesuai dengan karakteristik proyek berskala kecil hingga menengah, serta memungkinkan penyesuaian fitur berdasarkan hasil evaluasi pada setiap tahap pengembangan.

Pendekatan ini memungkinkan sistem dikembangkan secara inkremental, di mana setiap iterasi menghasilkan fungsionalitas yang dapat diuji dan dievaluasi sebelum melanjutkan ke iterasi berikutnya. Dengan demikian, risiko kesalahan desain dan implementasi dapat diminimalkan sejak tahap awal.

\subsection{Konsep Agile Development}

Agile Development merupakan pendekatan pengembangan perangkat lunak yang berfokus pada fleksibilitas, kolaborasi, dan kemampuan beradaptasi terhadap perubahan kebutuhan \cite{ariesta2020}. Berbeda dengan metode linear seperti waterfall, Agile memungkinkan perubahan kebutuhan terjadi selama proses pengembangan tanpa harus mengulang seluruh tahapan dari awal.

Dalam konteks pengembangan platform Nyumbangin, Agile digunakan sebagai kerangka kerja konseptual untuk mengelola proses pengerjaan fitur secara bertahap, mulai dari analisis kebutuhan dasar, implementasi modul inti, hingga pengujian dan evaluasi sistem. Pendekatan ini mendukung pengembangan sistem yang responsif terhadap kebutuhan fungsional dan teknis yang berkembang selama proyek berlangsung.

\subsection{Alur Iterasi Pengembangan}

Proses pengembangan sistem dilakukan melalui beberapa siklus iterasi yang masing-masing mencakup tahapan berikut:

\begin{enumerate}
    \item \textbf{Perencanaan Iterasi} \\
    Penentuan fitur yang akan dikembangkan berdasarkan prioritas kebutuhan sistem.

    \item \textbf{Implementasi Fitur} \\
    Pengembangan modul atau fungsi tertentu sesuai dengan hasil perencanaan iterasi.

    \item \textbf{Pengujian Fungsional} \\
    Pengujian terhadap fitur yang telah dikembangkan untuk memastikan kesesuaian dengan kebutuhan.

    \item \textbf{Evaluasi dan Penyempurnaan} \\
    Evaluasi hasil iterasi dan perbaikan terhadap kukurangan sebelum melanjutkan ke iterasi berikutnya.
\end{enumerate}

Setiap iterasi menghasilkan peningkatan fungsional sistem yang dapat langsung diuji, sehingga kemajuan proyek dapat dipantau secara berkelanjutan.

\subsection{Penerapan Agile pada Proyek Nyumbangin}

Penerapan pendekatan Agile pada proyek Nyumbangin dilakukan dengan membagi pengembangan sistem ke dalam beberapa iterasi utama. Iterasi awal difokuskan pada pembangunan fitur inti, seperti autentikasi kreator, pencatatan donasi, dan integrasi sistem pembayaran. Iterasi berikutnya mencakup pengembangan fitur pendukung, seperti notifikasi real-time melalui \textit{overlay}, \textit{leaderboard} donatur, serta mekanisme \textit{payout} bagi kreator.

Pada setiap iterasi, fitur yang telah diimplementasikan langsung diuji menggunakan scenario pengujian fungsional untuk memastikan alur donasi, pembayaran, dan pencairan dana berjalan dengan benar. Hasil pengujian digunakan sebagai dasar evaluasi untuk menentukan perbaikan atau pengembangan fitur pada iterasi selanjutnya.

Pendekatan ini memungkinkan pengembangan aplikasi dilakukan secara terstruktur namun tetap fleksibel, sehingga sistem dapat berkembang secara bertahap hingga memenuhi kebutuhan fungsional yang telah ditetapkan.

\section{Analisis Kebutuhan}

\subsection{Sumber Kebutuhan}

Analisis kebutuhan sistem dilakukan melalui tiga pendekatan utama. Pertama, observasi terhadap platform donasi digital untuk mengidentifikasi pola umum, seperti kebutuhan transparansi transaksi, tampilan notifikasi real-time, dan mekanisme payout yang akuntabel. Kedua, studi pustaka terkait autentikasi modern (OAuth 2.0, JWT), keamanan API, serta pola desain sistem web yang relevan dengan karakteristik aplikasi donasi. Ketiga, analisis artefak kode dan struktur API yang telah dikembangkan, termasuk model basis data, alur donasi, integrasi Midtrans, serta skrip pemeliharaan yang digunakan untuk verifikasi dan pengarsipan data. Pendekatan ini memastikan kebutuhan sistem dirumuskan berdasarkan konteks teknis dan operasional yang aktual.

\subsection{Kebutuhan Fungsional}

Kebutuhan fungsional mendeskripsikan fitur yang wajib disediakan agar aplikasi donasi dapat berfungsi secara utuh. Fitur tersebut meliputi:

\begin{enumerate}
    \item Autentikasi Pengguna menggunakan Google OAuth melalui NextAuth, serta validasi akses melalui JWT untuk endpoint sensitif.
    \item Validasi Username Kreator, memastikan username unik dan dapat diverifikasi sebelum transaksi dilakukan.
    \item Pengelolaan Donasi, mencakup pembuatan transaksi, integrasi Midtrans, pembaruan status melalui webhook, serta penyimpanan konten media share.
    \item Notifikasi Real-Time, yang menampilkan donasi terbaru pada overlay kreator untuk keperluan siaran langsung.
    \item Leaderboard, baik secara global maupun berdasarkan kreator, untuk menampilkan agregasi donasi.
    \item Statistik Kreator, berupa ringkasan donasi berdasarkan periode.
    \item Payout, mencakup permintaan penarikan dana, perhitungan biaya platform, dan persetujuan admin.
    \item Pengelolaan Data Media Share dan Notifikasi, termasuk pengaturan masa berlaku (TTL) dan keterkaitan dengan transaksi donasi.
\end{enumerate}

\subsection{Kebutuhan Non-Fungsional}

Kebutuhan non-fungsional mencakup karakteristik kualitas sistem, yaitu:

\begin{itemize}
    \item \textbf{Keamanan}: verifikasi token JWT, sanitasi input, pembatasan metode HTTP, dan pemisahan akses berdasarkan peran pengguna.
    \item \textbf{Performa}: optimasi query leaderboard melalui limit dan sorting.
    \item \textbf{Skalabilitas}: rencana pagination serta caching pada proses agregasi data.
    \item \textbf{Reliabilitas}: konsistensi penanganan kesalahan dengan kode status standar (401, 404, 500).
    \item \textbf{Integritas Data}: akurasi perhitungan saldo payout dan pemrosesan donasi berdasarkan statur valid (PAID).
\end{itemize}

\section{Perancangan Sistem}

\subsection{Arsitektur Logis}

Arsitektur logis sistem terdiri atas empat lapisan utama sebagai berikut:

\begin{enumerate}
    \item \textbf{Lapisan Antarmuka Pengguna (UI Layer)} \\
    Berisi halaman donasi, dashboard kreator, halaman overlay notifikasi, serta antarmuka admin.

    \item \textbf{Lapisan API (Application Layer)} \\
    Mengelola endpoint untuk donasi, \textit{leaderboard}, \textit{overlay}, autentikasi, \textit{payout}, dan operasi admin melalui mekanisme \textit{API} Routes di Next.js.

    \item \textbf{Lapisan Layanan Utilitas (Service Layer)} \\
    Meliputi modul koneksi database, pengelolaan token JWT, serta utilitas untuk validasi dan perhitungan internal.

    \item \textbf{Lapisan Data (Data Layer)} \\
    Terdiri atas model MongoDB seperti Donation, Creator, MediaShare, Payout, dan Notification.
\end{enumerate}

\subsection{Arsitektur Fisik}

Arsitektur fisik sistem mengikuti pola aplikasi web modern sebagai berikut:

\begin{center}
Browser / OBS Overlay $\rightarrow$ Next.js Runtime (Node.js) $\rightarrow$ MongoDB $\rightarrow$ Layanan Eksternal (Google OAuth, Midtrans)
\end{center}

\begin{itemize}
    \item Next.js menangani logika antarmuka pengguna dan API dalam satu platform.
    \item MongoDB digunakan sebagai basis data dokumen.
    \item Midtrans mengelola proses pembayaran melalui mekanisme \textit{Snap Token} dan \textit{webhook}.
    \item Overlay digunakan secara mandiri melalui OBS atau \textit{iframe} untuk menampilkan notifikasi donasi.
\end{itemize}

\subsection{Arsitektur Teknologi}

Arsitektur teknologi sistem mencakup penggunaan Next.js sebagai framework utama yang menjalankan frontend dan backend melalui API Routes, Node.js sebagai runtime server-side, serta MongoDB sebagai basis data dokumen. Sistem autentikasi menggunakan Google OAuth 2.0 melalui NextAuth dan JWT untuk otorisasi endpoint privat. Mekanisme pembayaran dilakukan dengan Midtrans melalui Snap Token dan Webhook. Selain itu, aplikasi menyediakan overlay web real-time untuk integrasi dengan OBS sebagai tampilan notifikasi donasi. Kombinasi teknologi ini menghasilkan sistem yang modern, responsive, serta siap diintegrasikan dengan berbagai layanan eksternal.

\begin{figure}[H]
    \centering
    \includegraphics[width=\textwidth]{images/arsitektur.png}
    \caption{Arsitektur Teknologi}
    \label{fig:arsitektur-teknologi}
\end{figure}

\subsection{Modul Utama}

Sistem dibagi ke dalam beberapa modul utama sebagai berikut:

\begin{enumerate}
    \item \textbf{Authentication Module} (OAuth + JWT)
    \item \textbf{Donation Module} (pembuatan transaksi, webhook, media share)
    \item \textbf{Leaderboard Module} (global dan per kreator)
    \item \textbf{Payout Module} (request, approval, perhitungan fee)
    \item \textbf{Notification Module} (TTL, event donasi)
    \item \textbf{Maintenance Module} (arsip donasi, integritas data)
\end{enumerate}

\subsection{Strategi Desain}

Strategi desain menerapkan pola penanganan \textit{API} yang konsisten meliputi validasi metode HTTP, autentikasi, validasi input, eksekusi query database, dan pengembalian response dalam format JSON. Selain itu, prinsip pemisahan tanggung jawab diterapkan melalui pembagian endpoint berdasarkan peran dan fungsi. Pembatasan data seperti penggunaan limit dan sorting digunakan untuk menghindari over-fetching, terutama pada proses pengambilan data leaderboard dan statistik.

\section{Pemodelan Sistem}

Pemodelan sistem dilakukan untuk memberikan representasi visual dan konseptual dari fungsi, alur kerja, serta struktur data yang digunakan dalam aplikasi Nyumbangin ini. Pemodelan ini bertujuan memastikan bahwa kebutuhan fungsional dan non-fungsional yang telah diidentifikasi dapat diterjemahkan ke dalam desain sistem yang jelas, terstruktur, dan mudah diimplementasikan. Diagram-diagram pada bagian ini mencakup model proses, interaksi, dan entitas yang saling berhubungan, sehingga dapat memberikan gambaran menyeluruh mengenai cara sistem beroperasi secara end-to-end.

\subsection{Use Case}

\begin{figure}[H]
    \centering
    \includegraphics[width=\textwidth]{images/use-case-diagram.png}
    \caption{Use Case Diagram}
    \label{fig:usecase}
\end{figure}

Use case menggambarkan interaksi antara aktor dengan sistem aplikasi Nyumbangin. Diagram ini bertujuan untuk menunjukkan fungsi-fungsi utama yang dapat dilakukan oleh masing-masing aktor sesuai dengan perannya.

Aktor yang terlibat dalam sistem ini yaitu Donatur, Kreator, dan Admin. Selain itu, sistem terintegrasi dengan layanan eksternal seperti Google OAuth untuk proses autentikasi dan Midtrans sebagai gateway pembayaran. Use case ini menggambarkan secara umum cakupan fungsi sistem tanpa menampilkan detail alur proses.

\subsection{Activity Diagram}

\textbf{Activity Diagram proses Donasi}

\begin{figure}[H]
    \centering
    \includegraphics[width=\textwidth]{images/activity-diagram-donasi.png}
    \caption{Activity Diagram Donasi}
    \label{fig:activity-donasi}
\end{figure}

Activity diagram proses donasi menggambarkan alur di mana Donor mengisi formulir donasi (dengan atau tanpa media share), kemudian sistem membuat record donasi dengan status \texttt{PENDING}, menghasilkan \textit{snap token}, dan mengarahkan Donor ke Midtrans untuk proses pembayaran. Setelah Midtrans mengirimkan webhook, sistem memperbarui status donasi menjadi \texttt{PAID} dan membuat notifikasi donasi.

\subsection{Activity Diagram Proses Payout}

\begin{figure}[H]
    \centering
    \includegraphics[width=\textwidth]{images/activity-diagram-payout.png}
    \caption{Activity Diagram Payout}
    \label{fig:activity-payout}
\end{figure}

Activity diagram proses payout dimulai ketika Kreator mengajukan permintaan payout, di mana sistem melakukan pemeriksaan terhadap saldo minimal. Apabila saldo memenuhi syarat, sistem akan menghitung biaya layanan sebesar 5\% dan mengubah status payout menjadi \texttt{PENDING}. Selanjutnya, Admin melakukan peninjauan dan transfer dana secara manual, setelah itu sistem memperbarui status payout menjadi \texttt{PROCESSED}.

\subsection{Activity Diagram Feedback}

\begin{figure}[H]
    \centering
    \includegraphics[width=\textwidth]{images/activity-diagram-feedback.png}
    \caption{Activity Diagram Feedback}
    \label{fig:activity-feedback}
\end{figure}

Activity diagram feedback menggambarkan alur proses pengiriman feedback oleh pengguna. Proses dimulai dari pengisian formulir feedback, kemudian sistem melakukan validasi data. Apabila data dinyatakan valid, sistem akan menyimpan feedback dan menampilkan konfirmasi bahwa feedback berhasil dikirim. Sebaliknya, jika data tidak valid, sistem akan menampilkan pesan kesalahan.

\subsection{Activity Diagram Login}

\begin{figure}[H]
    \centering
    \includegraphics[width=\textwidth]{images/activity-diagram-login.png}
    \caption{Activity Diagram Login}
    \label{fig:activity-login}
\end{figure}

Activity diagram login menggambarkan alur autentikasi pengguna menggunakan Google OAuth. Proses dimulai dari login, autentikasi ke Google, pengecekan data pengguna, pembuatan token akses, dan pengalihan ke halaman dashboard sesuai dengan peran pengguna.

\subsection{Activity Diagram Kelola Overlay}

\begin{figure}[H]
    \centering
    \includegraphics[width=\textwidth]{images/activity-diagram-kelola-overlay.png}
    \caption{Activity Diagram Kelola Overlay}
    \label{fig:activity-kelola-overlay}
\end{figure}

Activity diagram kelola overlay menggambarkan alur pengelolaan overlay oleh kreator. Sistem melakukan pengecekan autentikasi sebelum menampilkan halaman overlay. Kreator dapat menambah, mengubah, atau menghapus data overlay. Sistem memvalidasi dan menyimpan perubahan, kemudian memberikan feedback keberhasilan kepada kreator.

\subsection{Activity Diagram Kelola Profil}

\begin{figure}[H]
    \centering
    \includegraphics[width=\textwidth]{images/activity-diagram-kelola-profil.png}
    \caption{Activity Diagram Kelola Profil}
    \label{fig:activity-kelola-profil}
\end{figure}

Activity diagram kelola profil menggambarkan proses perubahan data profil oleh kreator. Kreator mengisi form perubahan data dan mengunggah foto profil jika diperlukan. Sistem melakukan validasi input dan menyimpan perubahan ke database. Jika proses berhasil, sistem menampilkan pesan sukses.

\subsection{Activity Diagram Request Payout}

\begin{figure}[H]
    \centering
    \includegraphics[width=\textwidth]{images/activity-diagram-request-payout.png}
    \caption{Activity Diagram Request Payout}
    \label{fig:activity-request-payout}
\end{figure}

Activity diagram request payout menggambarkan alur pencairan dana yang diajukan oleh kreator. Sistem memeriksa saldo dan minimum payout, menyimpan data payout, serta mengirimkan notifikasi ke admin. Admin melakukan proses review, dan sistem memperbarui status payout sesuai dengan keputusan admin.

\subsection{Activity Diagram Riwayat Donasi}

\begin{figure}[H]
    \centering
    \includegraphics[width=\textwidth]{images/activity-diagram-riwayat-donasi.png}
    \caption{Activity Diagram Riwayat Donasi}
    \label{fig:activity-riwayat-donasi}
\end{figure}

Activity diagram riwayat donasi menggambarkan proses penampilan data riwayat donasi. Kreator membuka halaman riwayat donasi, sistem menampilkan data donasi yang tersimpan, dan kreator dapat melakukan pencairan serta melakukan filter data berdasarkan kriteria tertentu.

\subsection{Sequence Diagram}

\subsubsection{Sequence Diagram Donasi}

\begin{figure}[H]
    \centering
    \includegraphics[width=\textwidth]{images/sequence-diagram-donasi.png}
    \caption{Sequence Diagram Donasi}
    \label{fig:sequence-donasi}
\end{figure}

Sequence diagram donasi menggambarkan alur proses donasi yang dimulai ketika donor mengisi formulir pada halaman donasi (\textit{Donation Page}) yang kemudian divalidasi dan dikirim ke \textit{Donate API}. Setelah dilakukan validasi di sisi server, data donasi disimpan ke dalam basis data, dan apabila terdapat \textit{youtubeUrl}, sistem juga akan membuat data \textit{MediaShare}. Selanjutnya, \textit{API} meminta \textit{Snap Token} ke Midtrans yang digunakan untuk membuka halaman pembayaran. Setelah donor menyelesaikan pembayaran dan transaksi dinyatakan berhasil, Midtrans mengirimkan webhook ke server sehingga status donasi diperbarui menjadi \texttt{PAID}, dan pada tahap akhir sistem membuat notifikasi overlay.

\subsubsection{Sequence Diagram Leaderboard}

\begin{figure}[H]
    \centering
    \includegraphics[width=\textwidth]{images/sequence-diagram-leaderboard.png}
    \caption{Sequence Diagram Leaderboard}
    \label{fig:sequence-leaderboard}
\end{figure}

Sequence diagram leaderboard menggambarkan alur pengambilan data \textit{leaderboard}, di mana client (\textit{dashboard} kreator) mengirimkan request ke \textit{Leaderboard API} dengan menyertakan token. API melakukan pemeriksaan metode HTTP dan validasi token melalui layanan JWT, kemudian memastikan bahwa pengguna merupakan kreator dan data yang diminta tersedia. Selanjutnya, API mengambil data donasi terbaru dari basis data, memformat respons, dan mengirimkan hasilnya kembali ke client. Apabila terjadi kesalahan seperti token tidak valid, kreator tidak ditemukan, atau metode HTTP tidak sesuai, API akan mengembalikan kode error yang relevan.

\subsubsection{Sequence Diagram Login}

\begin{figure}[H]
    \centering
    \includegraphics[width=\textwidth]{images/sequence-diagram-login.png}
    \caption{Sequence Diagram Login}
    \label{fig:sequence-login}
\end{figure}

Sequence diagram login menggambarkan urutan interaksi antara pengguna dan sistem dalam proses autentikasi menggunakan email dan kata sandi. Proses dimulai ketika pengguna membuka halaman login dan memasukkan data email serta kata sandi. Sistem kemudian melakukan pengecekan data ke basis data untuk memverifikasi kecocokan informasi yang dimasukkan. Apabila data tidak sesuai, sistem akan menampilkan pesan kesalahan kepada pengguna. Jika data sesuai, sistem memproses autentikasi dan mengarahkan pengguna ke halaman utama atau dashboard sesuai dengan hak akses yang dimiliki.

\subsubsection{Sequence Diagram Feedback}

\begin{figure}[H]
    \centering
    \includegraphics[width=\textwidth]{images/sequence-diagram-feedback.png}
    \caption{Sequence Diagram Feedback}
    \label{fig:sequence-feedback}
\end{figure}

Sequence diagram feedback menggambarkan alur pengiriman feedback dari pengguna ke sistem. Proses dimulai ketika pengguna membuka halaman kontak dan mengisi form feedback. Sistem melakukan validasi terhadap data yang dikirimkan. Jika data tidak valid, sistem akan mengirimkan pesan error. Jika data valid, sistem menyimpan feedback ke dalam basis data dan mengirimkan respons bahwa feedback berhasil dikirim.

\subsubsection{Sequence Diagram Kelola Profil}

\begin{figure}[H]
    \centering
    \includegraphics[width=\textwidth]{images/sequence-diagram-kelola-profil.png}
    \caption{Sequence Diagram Kelola Profil}
    \label{fig:sequence-kelola-profil}
\end{figure}

Sequence diagram kelola profil menggambarkan interaksi antara kreator dan sistem dalam memproses pengelolaan data profil. Kreator membuka menu profil dan melakukan perubahan data. Sistem menerima perubahan, melakukan validasi, dan menyimpan perubahan ke basis data. Setelah proses berhasil, sistem mengirimkan respons bahwa profil berhasil diperbarui.

\subsubsection{Sequence Diagram Riwayat Donasi}

\begin{figure}[H]
    \centering
    \includegraphics[width=\textwidth]{images/sequence-diagram-riwayat-donasi.png}
    \caption{Sequence Diagram Riwayat Donasi}
    \label{fig:sequence-riwayat-donasi}
\end{figure}

Sequence diagram riwayat donasi menggambarkan proses pengambilan dan penampilan data riwayat donasi. Kreator mengakses halaman riwayat donasi, sistem melakukan validasi autentikasi, kemudian mengambil data donasi dari basis data. Kreator dapat melakukan pencarian dan filter data, dan sistem akan menampilkan hasil sesuai dengan permintaan.

\subsubsection{Sequence Diagram Request Payout}

\begin{figure}[H]
    \centering
    \includegraphics[width=\textwidth]{images/sequence-diagram-request-payout.png}
    \caption{Sequence Diagram Request Payout}
    \label{fig:sequence-request-payout}
\end{figure}

Sequence diagram request payout menggambarkan proses pengajuan pencairan dana oleh kreator. Kreator mengajukan permintaan payout melalui sistem. Sistem melakukan pengecekan saldo dan menyimpan data payout dengan status \texttt{PENDING}. Admin kemudian melakukan proses review terhadap permintaan tersebut dan memutuskan untuk menerima atau menolak. Apabila permintaan diterima, sistem memperbarui status payout dan mengirimkan notifikasi keberhasilan kepada kreator.

\subsection{Model Koleksi}

Model koleksi digunakan untuk merepresentasikan struktur data utama yang bekerja di dalam platform donasi. Setiap collection dirancang untuk mendukung proses transaksi, pengelolaan kreator, penayangan media share di overlay, hingga alur pancairan dana. Secara umum, entitas yang digunakan dapat dikelompokkan menjadi koleksi utama dan koleksi pendukung. 

Koleksi utama platform meliputi:

\begin{itemize}
    \item \textbf{Creator} \\
    Menyimpan data kreator seperti nama, email, profil, serta informasi akun yang diperlukan untuk menerima donasi dan melakukan permintaan payout.

    \item \textbf{Donation} \\
    Mencatat seluruh transaksi donasi, termasuk nominal, pesan, metode pembayaran, status (\texttt{PENDING}/\texttt{PAID}), serta relasi terhadap kreator yang menerima donasi.

    \item \textbf{MediaShare} \\
    Entitas untuk menangani request media share (YouTube video) yang dikaitkan dengan donasi tertentu, termasuk durasi dan validasi media.

    \item \textbf{Payout} \\
    Menyimpan informasi permintaan pencairan dana kreator, mencakup jumlah pencairan, fee platform, status (\texttt{PENDING}/\texttt{APPROVED}/\texttt{PROCESSED}), serta log aktivitas admin.

    \item \textbf{Notification} \\
    Berfungsi untuk menampilkan data overlay secara real-time, seperti donasi terbaru atau media share yang harus ditayangkan oleh streamer atau kreator.
\end{itemize}

Selain itu, terdapat entitas pendukung yang digunakan untuk historisasi dan agregasi data, yaitu:

\begin{itemize}
    \item \textbf{DonationHistory} – mencatat perubahan status donasi.
    \item \textbf{MonthlyLeaderboard} – menyimpan data peringkat donatur setiap bulan.
    \item \textbf{Contact} – mencatat umpan balik dari pengguna.
    \item \textbf{ProfileImage} – menyimpan data gambar untuk kebutuhan profil kreator.
    \item \textbf{Filteredwords} – menyimpan daftar kata yang difilter pada pesan donasi.
    \item \textbf{Donationshares} – menyimpan \textit{record} data agregasi dari hasil \textit{sharelink} donasi.
    \item \textbf{Admin} – menyimpan kredensial admin yang bertugas memverifikasi payout dan melakukan manajemen sistem.
\end{itemize}

Seluruh koleksi tersebut berperan dalam memastikan integritas data serta menghubungkan seluruh proses inti mulai dari transaksi donasi, pengelolaan kreator, hingga operasional sistem admin.

\subsection{Gopay Merchant}

GoPay Merchant digunakan sebagai metode pembayaran alternatif pada aplikasi Nyumbangin selain integrasi utama melalui Midtrans. Fitur ini memungkinkan donatur melakukan pembayaran dengan memindai kode QRIS yang disediakan melalui aplikasi merchant. Setiap transaksi yang berhasil akan menghasilkan notifikasi pada perangkat admin sebagai indikator penerimaan dana. Implementasi ini bertujuan untuk memberikan fleksibilitas metode pembayaran, khususnya untuk transaksi langsung atau kebutuhan tertentu di luar alur payment gateway utama, tanpa menggantikan sistem pembayaran terintegrasi yang telah dibangun sebelumnya.

\subsection{Perancangan Mekanisme Macrodroid Trigger Webhook}

MacroDroid digunakan sebagai mekanisme otomasi untuk mendeteksi notifikasi transaksi dari aplikasi GoPay Merchant dan meneruskannya ke sistem backend Nyumbangin. Aplikasi ini dikonfigurasi menggunakan konsep trigger dan action, di mana trigger berupa notifikasi transaksi masuk, kemudian action yang dijalankan adalah pengiriman HTTP request ke endpoint backend yang telah disediakan. Data yang diterima backend selanjutnya diproses dan disimpan ke dalam database sehingga donasi dapat tercatat secara otomatis tanpa input manual dari admin. Pendekatan ini memungkinkan integrasi pembayaran berjalan secara semi-otomatis dengan memanfaatkan sistem notifikasi pada perangkat Android.

\begin{figure}[H]
    \centering
    \includegraphics[width=\textwidth,height=0.8\textheight,keepaspectratio]{images/macrodroid.jpg}
    \caption{MacroDroid}
    \label{fig:macrodroid}
\end{figure}

\subsection{Integrasi Payment Gateway Midtrans}

Midtrans dirancang sebagai payment gateway yang menangani proses pembayaran donasi secara eksternal. Sistem Nyumbangin hanya membuat transaksi awal dan menghasilkan Snap Token berdasarkan data donasi yang telah divalidasi. Snap Token tersebut digunakan untuk menampilkan halaman pembayaran Midtrans, sementara penentuan status akhir transaksi tidak dilakukan pada tahap ini dan dibahas pada mekanisme selanjutnya. Implementasi tampilan pembayaran Midtrans ditampilkan pada BAB IV.

\section{Metode Perancangan Teknis}

Perancangan teknis pada platform donasi ini difokuskan pada penyusunan arsitektur layanan yang aman, efisien, dan mudah dipelihara. Pendekatan utama yang digunakan adalah pemisahan tanggung jawab antar modul serta penerapan pola penanganan API yang konsisten. Setiap endpoint dirancang mengikuti alur standar, yaitu validasi metode HTTP, autentikasi menggunakan JWT untuk endpoint privat, validasi input, eksekusi operasi basis data, dan pengembalian respons dalam format JSON. Pola yang seragam ini memudahkan proses debugging serta menjaga konsistensi perilaku lintas layanan.

Dari sisi keamanan, validasi token JWT diterapkan untuk memastikan setiap permintaan memiliki otorisasi yang benar, termasuk pengecekan masa berlaku token dan jenis pengguna (donor, kreator, atau admin). Seluruh input kritis seperti username, nominal donasi, dan URL media share divalidasi untuk mencegah data tidak sah masuk ke dalam sistem. Integrasi pembayaran dirancang agar hanya bergantung pada webhook resmi Midtrans, sehingga status transaksi tidak bergantung pada aktivitas pengguna di sisi klien.

Optimasi basis data dilakukan melalui penempatan indeks pada atribut yang sering digunakan dalam kueri, seperti \texttt{createdAt}, \texttt{creatorId}, dan \texttt{creatorUsername}. Operasi agregasi seperti \textit{leaderboard} dan statistik kreator menggunakan pipeline agregasi \textit{MongoDB} untuk mengurangi beban pada server aplikasi. Pembatasan kueri (\textit{limit}) diterapkan untuk mencegah pengambilan data berlebihan pada endpoint yang memiliki potensi pertumbuhan data tinggi.

Selain itu, proses donasi dan penyajian notifikasi dipisahkan dari alur pembayaran utama. Server hanya membuat record donasi dan \textit{Snap Token} pada permintaan awal, sementara pembaruan status dan pemicu notifikasi dilakukan ketika \textit{webhook} diterima. Pemisahan ini meningkatkan stabilitas sistem dan memastikan overlay hanya menampilkan data yang telah tervalidasi.

\section{Metode Pengujian}

Metode pengujian pada platform donasi ini dirancang untuk memastikan bahwa seluruh fungsi sistem berjalan sesuai kebutuhan, aman digunakan, dan menghasilkan data yang konsisten. Pendekatan pengujian dilakukan melalui kombinasi pengujian unit, pengujian integrasi, serta pengujian fungsional terhadap endpoint API dan alur bisnis utama. Fokus utama pengujian meliputi keakuratan proses donasi, keandalan mekanisme payout, integritas data pada leaderboard serta statistik kreator, dan validitas proses autentikasi berbasis OAuth dan JWT.

Pengujian dilakukan menggunakan data uji terkontrol, simulasi webhook pembayaran, serta verifikasi hasil secara langsung melalui basis data. Seluruh skenario kritis seperti validasi input, autentikasi dan otorisasi, serta penanganan kesalahan diuji untuk memastikan sistem tetap stabil dalam berbagai kondisi operasional.

\subsection{Jenis Pengujian}

Pengujian sistem mencakup beberapa jenis pengujian sebagai berikut:

\begin{enumerate}[label=\alph*.]
    \item \textbf{Pengujian Unit (Unit Testing)} \\
    Dilakukan pada fungsi atau modul kecil yang berdiri sendiri, seperti:
    \begin{itemize}
        \item Validasi token JWT (sign dan verify),
        \item Perhitungan \textit{platformFee} dan \textit{finalAmount} pada payout,
        \item Validasi input donasi (amount, username, media share).
    \end{itemize}

    \item \textbf{Pengujian Integrasi (Integration Testing)} \\
    Berfokus pada alur yang melibatkan beberapa komponen, seperti:
    \begin{itemize}
        \item Proses donasi lengkap (create donation $\rightarrow$ Snap Token $\rightarrow$ pembayaran $\rightarrow$ webhook $\rightarrow$ update status),
        \item Permintaan payout oleh kreator dan proses approval oleh admin,
        \item Penampilan notifikasi overlay berdasarkan data terbaru.
    \end{itemize}

    \item \textbf{Pengujian Fungsional (Functional Testing)} \\
    Dilakukan untuk mengevaluasi apakah setiap endpoint memenuhi kebutuhan fungsional yang telah ditetapkan, seperti:
    \begin{itemize}
        \item Validasi username saat login atau donasi,
        \item Pembatasan akses endpoint dashboard menggunakan JWT,
        \item Respons error 400, 401, 404, dan 500 yang konsisten.
    \end{itemize}

    \item \textbf{Pengujian Keamanan (Security Testing)} \\
    Meliputi:
    \begin{itemize}
        \item Akses API tanpa token harus ditolak,
        \item Token kedaluwarsa atau \textit{userType} tidak sesuai harus gagal,
        \item Percobaan pengiriman payload tidak valid harus tervalidasi.
    \end{itemize}

    \item \textbf{Pengujian Kinerja (Performance Testing)} \\
    Difokuskan pada:
    \begin{itemize}
        \item Kecepatan query leaderboard (target $< 300$ ms),
        \item Stabilitas respons webhook,
        \item Performa overlay saat proses polling data.
    \end{itemize}
\end{enumerate}

\subsection{Skenario Pengujian}

Beberapa skenario pengujian utama yang digunakan meliputi:

\begin{enumerate}
    \item \textbf{Skenario 1 – Donasi Berhasil} \\
    Input valid $\rightarrow$ server membuat record \texttt{PENDING} $\rightarrow$ Snap Token sukses $\rightarrow$ pembayaran di Midtrans $\rightarrow$ webhook diterima $\rightarrow$ status menjadi \texttt{PAID} $\rightarrow$ overlay menampilkan donasi.

    \item \textbf{Skenario 2 – Donasi Gagal Validasi} \\
    Nominal di bawah minimum atau format URL salah $\rightarrow$ server mengembalikan status 400.

    \item \textbf{Skenario 3 – Akses Endpoint Tanpa Token} \\
    Mengakses leaderboard atau payout tanpa header \textit{Authorization} $\rightarrow$ sistem mengembalikan 401 \textit{Unauthorized}.

    \item \textbf{Skenario 4 – Request Payout oleh Kreator} \\
    Saldo cukup $\rightarrow$ request dicatat $\rightarrow$ status \texttt{PENDING} $\rightarrow$ admin review $\rightarrow$ \texttt{APPROVED} $\rightarrow$ status \texttt{PROCESSED} $\rightarrow$ saldo kreator berkurang sesuai \textit{finalAmount}.

    \item \textbf{Skenario 5 – Token Salah atau Kedaluwarsa} \\
    Token tidak valid atau kedaluwarsa $\rightarrow$ akses ditolak dengan pesan error yang konsisten.

    \item \textbf{Skenario 6 – Webhook Simulasi Midtrans} \\
    Webhook dikirim manual dari Postman $\rightarrow$ status donasi berubah menjadi \texttt{PAID} $\rightarrow$ overlay menampilkan notifikasi.
\end{enumerate}

\subsection{Alat Pengujian}

Alat yang digunakan dalam proses pengujian meliputi:

\begin{itemize}
    \item \textbf{Postman} – untuk pengujian API (header, body, autentikasi).
    \item \textbf{MongoDB Compass} – untuk memverifikasi perubahan data secara langsung.
    \item \textbf{Logging Next.js (console)} – untuk memantau webhook, error, dan alur proses.
\end{itemize}

\section{Evaluasi Keberhasilan}

Evaluasi keberhasilan dilakukan untuk menilai sejauh mana implementasi sistem memenuhi kebutuhan fungsional, stabilitas operasional, serta ketepatan mekanisme kritis seperti autentikasi, pengelolaan sesi, dan pemrosesan pembayaran. Penilaian dilakukan melalui pengujian terstruktur dan analisis hasil \textit{code coverage} yang dihasilkan dari proses unit testing pada modul-modul inti.

\subsection{Cakupan Pengujian}

Pengujian difokuskan pada komponen yang tergolong kritikal bagi kelangsungan sistem, meliputi:

\begin{itemize}
    \item Autentikasi dan Otorisasi (JWT, session handling, route protection)
    \item Pemrosesan Pembayaran (Midtrans Snap dan webhook)
    \item Fungsi donasi (validasi input, perhitungan status)
    \item Fungsi payout (perhitungan \textit{finalAmount}, status machine payout)
\end{itemize}

Komponen lain seperti UI Component dan utilitas pendukung tidak menjadi prioritas utama dalam pengujian ini karena tidak langsung berpengaruh pada keamanan maupun alur transaksi.

\subsection{Hasil Code Coverage}

Pengujian menghasilkan metrik \textit{code coverage} sebagai berikut:

\begin{table}[H]
\centering
\caption{Code Coverage}
\label{tab:code-coverage}
\begin{tabular}{|l|c|}
\hline
\textbf{Metode} & \textbf{Cakupan} \\ \hline
Statements & 14.71\% \\ \hline
Branch & 15.56\% \\ \hline
Functions & 29.35\% \\ \hline
Lines & 14.21\% \\ \hline
\end{tabular}
\end{table}

Meskipun angka coverage terlihat rendah secara keseluruhan, hal ini tidak mencerminkan kualitas fungsional sistem secara langsung karena coverage tidak mencakup seluruh file, melainkan hanya modul-modul yang dipilih berdasarkan kategori kritis. File non-kritis seperti UI, utilitas ringan, dan helper statis tidak disertakan dalam pengujian sehingga turut menurunkan total persentase.

Hasil tersebut menegaskan bahwa:

\begin{itemize}
    \item Seluruh proses inti (autentikasi, sesi, validasi donasi, perhitungan payout, dan pembayaran) telah berhasil diuji.
    \item Jalur eksekusi utama (\textit{main happy path}) telah tervalidasi.
    \item Beberapa jalur error belum sepenuhnya dicakup, sehingga membuka peluang peningkatan lebih lanjut.
\end{itemize}

\subsection{Interpretasi dan Evaluasi}

Berdasarkan hasil pengujian dan \textit{code coverage} tersebut, dapat disimpulkan bahwa:

\begin{enumerate}
    \item Stabilitas alur bisnis utama telah terverifikasi, terutama mekanisme donasi dan payout yang melibatkan transaksi dan webhook.
    \item Keamanan dasar terkait autentikasi, JWT, dan sesi telah diuji dan berfungsi sesuai kebutuhan.
    \item Konsistensi data pada proses pembayaran serta pencatatan notifikasi berhasil diuji melalui simulasi webhook.
    \item Rendahnya nilai coverage lebih disebabkan oleh fokus pengujian pada modul kritis, bukan karena seluruh sistem tidak diuji.
    \item Sistem dinilai layak digunakan, namun peningkatan cakupan pengujian tetap direkomendasikan untuk modul non-kritis seperti UI dan utilitas.
\end{enumerate}

\subsection{Kesimpulan Evaluasi}

Secara keseluruhan, sistem telah memenuhi fungsi utamanya mulai dari pemrosesan donasi, pembayaran, hingga payout dan notifikasi. Hasil pengujian menunjukkan bahwa fitur inti berjalan stabil, meskipun pengujian yang lebih luas masih diperlukan pada tahap pengembangan selanjutnya.
