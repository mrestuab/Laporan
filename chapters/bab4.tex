\chapter{Hasil dan Pembahasan}

\section{Halaman Autentikasi}

\begin{figure}[H]
    \centering
    \includegraphics[width=\textwidth]{images/halaman-login.png}
    \caption{Halaman Login}
    \label{fig:halaman-login}
\end{figure}

Halaman autentikasi menampilkan dua opsi untuk login, yaitu login manual dan login menggunakan Google OAuth untuk mempermudah proses autentikasi pengguna.

\section{Halaman Donasi}

\subsection{Form Donasi}

\begin{figure}[H]
    \centering
    \includegraphics[width=\textwidth]{images/form-donasi.png}
    \caption{Form Donasi}
    \label{fig:form-donasi}
\end{figure}

Halaman donasi menyediakan input seperti nama donor, nominal donasi, pesan, serta opsi media share. Sistem melakukan validasi dasar terhadap data yang dimasukkan sebelum melanjutkan proses pembayaran ke Midtrans.

\subsection{Pembayaran Donasi Menggunakan QRIS Gopay Merchant}

\begin{figure}[H]
    \centering
    \includegraphics[width=\textwidth]{images/qris-gopay.png}
    \caption{QRIS Gopay Merchant}
    \label{fig:qris-gopay}
\end{figure}

Aplikasi Nyumbangin mengimplementasikan pembayaran donasi menggunakan QRIS GoPay Merchant untuk mempermudah donatur dalam melakukan transaksi digital. QRIS ditampilkan sebagai kode pembayaran resmi atas nama merchant Nyumbangin, Digital \& Kreatif, seperti ditunjukkan pada Gambar 4.3, dan digunakan oleh donatur dengan cara memindai kode QR melalui aplikasi pembayaran yang mendukung standar QRIS. Proses pembayaran dilakukan di luar sistem aplikasi, sementara konfirmasi keberhasilan transaksi ditentukan melalui mekanisme webhook yang memperbarui status donasi di sistem sebelum notifikasi dan proses lanjutan dijalankan.

\subsection{Halaman Pembayaran Midtrans}

\begin{figure}[H]
    \centering
    \includegraphics[width=\textwidth]{images/halaman-pembayaran-midtrans.png}
    \caption{Halaman Pembayaran Midtrans}
    \label{fig:pembayaran-midtrans}
\end{figure}

Setelah form donasi berhasil divalidasi, sistem mengirimkan permintaan \textit{Snap Token} ke Midtrans. Donor kemudian diarahkan ke halaman pembayaran yang disediakan oleh Midtrans untuk menyelesaikan proses transaksi.

\subsection{Fitur Sharelink Donasi}

\begin{figure}[H]
    \centering
    \includegraphics[width=\textwidth]{images/sharelink-donasi.png}
    \caption{Fitur Sharelink Donasi}
    \label{fig:sharelink-donasi}
\end{figure}

Setelah donasi berhasil dan status transaksi tervalidasi, sistem menampilkan dialog konfirmasi yang menyediakan fitur social sharing berupa tombol berbagi tautan donasi ke beberapa platform media sosial serta opsi menyalin tautan. Fitur ini bertujuan mendorong penyebaran link donasi secara organik dan meningkatkan keterlibatan sosial.

\section{Dashboard Kreator}

\begin{figure}[H]
    \centering
    \includegraphics[width=\textwidth]{images/dashboard-kreator.png}
    \caption{Dashboard Kreator}
    \label{fig:dashboard-kreator}
\end{figure}

Dashboard kreator menampilkan statistik utama seperti total donasi, total pendapatan, menu riwayat donasi, serta menu leaderboard donasi terbanyak per bulan.

\section{Halaman Request Payout}

\begin{figure}[H]
    \centering
    \includegraphics[width=\textwidth]{images/halaman-payout.png}
    \caption{Halaman Payout}
    \label{fig:halaman-payout}
\end{figure}

Kreator dapat mengajukan pencairan dana berdasarkan saldo yang tersedia. Sistem menghitung saldo bersih berdasarkan total donasi dengan status \texttt{PAID} yang belum pernah diproses dalam payout sebelumnya.

\section{Overlay}

Overlay berfungsi sebagai antarmuka yang ditampilkan pada platform streaming melalui OBS. Seluruh elemen \textit{overlay} diambil secara real-time dari \textit{API} sehingga kreator dapat menampilkan interaksi donasi secara langsung kepada penonton. Subbab ini menampilkan implementasi setiap komponen overlay.

\subsection{Overlay Notifikasi Donasi}

\begin{figure}[H]
    \centering
    \includegraphics[width=\textwidth]{images/overlay-notifikasi-donasi.png}
    \caption{Notifikasi Donasi}
    \label{fig:overlay-notifikasi}
\end{figure}

Overlay menampilkan notifikasi donasi secara real-time yang diambil melalui mekanisme polling API. Tampilan ini dikonfigurasi agar kompatibel dengan OBS untuk keperluan streaming.

\subsection{Overlay Media Share}

\begin{figure}[H]
    \centering
    \includegraphics[width=\textwidth]{images/overlay-media-share.png}
    \caption{Media Share}
    \label{fig:overlay-media-share}
\end{figure}

Overlay Media Share menampilkan video YouTube yang diputar berdasarkan permintaan donor. Pada tampilan ini, video muncul di area utama layar, sementara informasi donasi seperti nama donor, nominal, dan pesan singkat ditampilkan pada bagian bawah kiri layar.

\subsection{Overlay QR Link Donasi}

\begin{figure}[H]
    \centering
    \includegraphics[width=\textwidth]{images/overlay-qr-donasi.png}
    \caption{QR Link Donasi}
    \label{fig:overlay-qr-donasi}
\end{figure}

Overlay menyediakan kode QR statis yang mengarahkan penonton langsung ke halaman donasi kreator. QR ini memfasilitasi penonton untuk berdonasi tanpa harus mengetik tautan donasi secara manual.

\subsection{Overlay Leaderboard}

\begin{figure}[H]
    \centering
    \includegraphics[width=\textwidth]{images/overlay-leaderboard-donatur.png}
    \caption{Leaderboard Donatur}
    \label{fig:overlay-leaderboard}
\end{figure}

Overlay leaderboard menampilkan daftar pendukung terbesar (\textit{top donors}) atau peringkat total donasi yang telah dioptimasi menggunakan mekanisme limit dan sorting. Tampilan ini digunakan untuk memberikan apresiasi kepada donatur selama sesi streaming berlangsung.

\section{Dashboard Admin}

Bagian ini menampilkan antarmuka yang digunakan admin untuk mengelola aktivitas pada platform, mulai dari pemantauan data donasi hingga pengelolaan kreator dan proses payout. Dashboard admin terdiri dari tiga menu utama, yaitu Dashboard, Creator, dan Payout.

\subsection{Halaman Dashboard}

\begin{figure}[H]
    \centering
    \includegraphics[width=\textwidth]{images/dashboard-admin.png}
    \caption{Dashboard Admin}
    \label{fig:dashboard-admin}
\end{figure}

Halaman dashboard menampilkan ringkasan statistik utama sistem, seperti jumlah kreator terdaftar, jumlah pengajuan payout, serta jumlah payout yang telah diselesaikan. Selain itu, halaman ini juga memvisualisasikan Top 5 Kreator Paling Aktif berdasarkan total donasi yang diterima.

\subsection{Halaman Creator}

\begin{figure}[H]
    \centering
    \includegraphics[width=\textwidth]{images/daftar-creator-admin.png}
    \caption{Tabel Daftar Creator dan Detail Creator}
    \label{fig:creator-admin}
\end{figure}

Halaman creator digunakan untuk melihat daftar kreator yang terdaftar pada platform, termasuk informasi status kelengkapan data payout masing-masing. Admin dapat melakukan proses pencairan dana kreator serta melihat detail lengkap akun kreator melalui dialog \textit{Detail Creator}.

\subsection{Halaman Payout}

\begin{figure}[H]
    \centering
    \includegraphics[width=\textwidth]{images/halaman-payout-admin.png}
    \caption{Tampilan Payout}
    \label{fig:payout-admin}
\end{figure}

Halaman payout menampilkan daftar pengajuan pencairan dana yang dilakukan oleh para kreator. Admin dapat memantau nominal pencairan, waktu pengajuan, serta catatan yang terkait dengan setiap payout. Halaman ini membantu admin memastikan bahwa seluruh proses pancairan berjalan dengan transparan dan terdokumentasi.

\section{Struktur Basis Data}

\begin{figure}[H]
    \centering
    \includegraphics[width=\textwidth]{images/struktur-basis-data.png}
    \caption{Basis Data}
    \label{fig:strukur-basis-data}
\end{figure}

Sistem menggunakan MongoDB sebagai basis data utama. Setiap fitur pada sistem menghasilkan koleksi tersendiri untuk memudahkan proses penyimpanan, pemantauan, dan analisis data.

Koleksi yang terbentuk antara lain:

\begin{itemize}
    \item \textbf{admins} – menyimpan data akun admin yang memiliki akses panel pengelolaan.
    \item \textbf{contacts} – menyimpan data feedback atau saran dan keluhan dari pengguna.
    \item \textbf{creators} – menyimpan informasi kreator, termasuk data payout dan profil.
    \item \textbf{donations} – mencatat data donasi sementara yang masuk dari pengguna yang nantinya dialihkan ke koleksi donationhistories.
    \item \textbf{donationhistories} – menyimpan data riwayat donasi yang berasal dari koleksi donations.
    \item \textbf{donationshares} – mencatat data share link donasi.
    \item \textbf{filteredwords} – menyimpan data filter kata-kata yang tidak pantas.
    \item \textbf{mediashares} – menyimpan data media share berupa request video YouTube.
    \item \textbf{monthlyleaderboards} – menyimpan data peringkat donor per bulan.
    \item \textbf{notifications} – menyimpan data notifikasi donasi yang ditampilkan pada overlay.
    \item \textbf{payouts} – mencatat pengajuan dan proses pencairan dana oleh kreator.
    \item \textbf{profileimages} – menyimpan informasi gambar profil kreator.
\end{itemize}

