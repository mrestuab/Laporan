\section{Kesimpulan}

Proses pemulihan pasien rawat jalan maupun rehabilitasi membutuhkan aksesibilitas terhadap alat medis penunjang, namun hal ini seringkali terkendala oleh biaya dan ketersediaan stok di fasilitas kesehatan. Oleh karena itu, sistem informasi MedisLink menjadi langkah strategis untuk menghubungkan pasien dengan donatur yang memiliki alat medis yang tidak terpakai. 

Melalui platform berbasis web yang mengintegrasikan fitur peminjaman dan donasi secara terpusat, alat kesehatan seperti kursi roda dan tabung oksigen dapat didistribusikan secara lebih efisien, transparan, dan terverifikasi dibandingkan metode konvensional. Sistem ini tidak hanya mempermudah masyarakat dalam memperoleh bantuan medis, tetapi juga membangun ekosistem sosial yang saling mendukung melalui digitalisasi inventaris dan validasi identitas pengguna yang jelas.

\section{Saran}

Meskipun sistem saat ini telah berjalan secara stabil dengan arsitektur cloud yang terpisah antara frontend, backend, dan database, pengembangan lebih lanjut tetap diperlukan untuk meningkatkan responsivitas dan kualitas layanan. Integrasi fitur notifikasi secara \textit{real-time}, misalnya melalui WhatsApp Gateway, dapat memberikan kepastian status persetujuan peminjaman secara instan kepada pengguna.

Selain itu, pengembangan ke arah aplikasi mobile atau \textit{Progressive Web App} (PWA) akan meningkatkan fleksibilitas akses, mengingat dominasi penggunaan perangkat seluler di masyarakat. Penambahan fitur visualisasi data analitik bagi administrator serta pelacakan logistik berbasis peta digital juga dapat meningkatkan efektivitas monitoring distribusi alat. 

Dengan pengembangan tersebut, diharapkan MedisLink dapat menjadi platform yang lebih responsif, fleksibel, dan komprehensif dalam mendukung kebutuhan kesehatan masyarakat secara luas.
