\section{Landasan Teori}

\subsection{API}

REST API (\textit{Representational State Transfer API}) merupakan kerangka kerja yang memungkinkan layanan diakses di berbagai platform dan lingkungan, mendukung interoperabilitas, serta mematuhi standar web. REST API menjadi metode umum untuk mempublikasikan layanan di internet karena sifatnya yang tidak terikat pada platform tertentu dan kemudahan implementasinya \cite{banubakode2022}. RESTful API terdiri dari berbagai \textit{endpoint} yang menunjukkan fungsi spesifik dari suatu proses bisnis dan dapat diakses melalui protokol HTTP menggunakan metode seperti \textit{GET}, \textit{POST}, \textit{PUT}, dan \textit{DELETE}.

JSON (\textit{JavaScript Object Notation}) menjadi format komunikasi standar dalam REST API karena kesederhanaan dan kompatibilitas lintas platform \cite{kanvar2022}. JSON banyak digunakan dalam arsitektur mikro-layanan karena memungkinkan pengembangan aplikasi yang fleksibel dan cepat \cite{habib2019}.

\subsection{Node.js}

Node.js adalah lingkungan \textit{runtime} JavaScript yang memungkinkan pengembang menulis kode sisi server menggunakan JavaScript \cite{gurusamy2020}. Dibangun di atas mesin V8 milik Google Chrome, Node.js mampu menangani operasi I/O non-blocking secara efisien. Platform ini menggunakan model berbasis peristiwa dengan mekanisme \textit{callback}, sehingga mampu menangani ribuan koneksi secara bersamaan tanpa memerlukan \textit{threading} kompleks \cite{yee2019}. Node.js banyak digunakan oleh perusahaan besar seperti PayPal, LinkedIn, Medium, dan Netflix karena performanya yang tinggi \cite{gurusamy2020}.

\subsection{Go (Golang)}

Go adalah bahasa pemrograman sumber terbuka yang dikompilasi dengan performa tinggi dan sintaks ringkas. Go mendukung pemrograman konkuren melalui \textit{goroutines} dan \textit{channel}, serta memiliki sistem tipe yang kuat dan modularitas berbasis paket. Selain cepat dalam kompilasi, Go menyediakan \textit{garbage collection} dan refleksi sehingga efektif untuk membangun layanan backend berskala besar \cite{goDocumentation}.

\subsection{React.js}

React.js merupakan pustaka JavaScript untuk membangun antarmuka pengguna berbasis komponen yang modular dan efisien \cite{annaram2024}. Salah satu keunggulan React adalah penggunaan \textit{Virtual DOM}, yang memungkinkan pembaruan tampilan secara cepat tanpa merender ulang seluruh halaman. Hal ini menjadikan React cocok untuk aplikasi web yang interaktif dan responsif.

\subsection{JavaScript}

JavaScript adalah bahasa pemrograman tingkat tinggi yang digunakan untuk membuat halaman web menjadi interaktif \cite{wai2022}. Bahasa ini memungkinkan pembuatan komponen seperti validasi formulir, galeri gambar responsif, dan animasi. Selain berjalan di sisi klien melalui browser, JavaScript juga dapat dijalankan di sisi server menggunakan lingkungan seperti Node.js \cite{shoikhedbrod2023}.

\subsection{MongoDB}

MongoDB adalah basis data NoSQL berbasis dokumen yang menyimpan data dalam format BSON/JSON. MongoDB mendukung skalabilitas horizontal dan fleksibilitas skema, sehingga cocok untuk aplikasi yang berkembang pesat. Namun, fleksibilitas ini memerlukan perencanaan data yang baik agar tidak menimbulkan inkonsistensi \cite{mongodbDocumentation}.

\subsection{DaisyUI}

DaisyUI adalah \textit{component library} berbasis Tailwind CSS yang menyediakan berbagai komponen antarmuka siap pakai seperti tombol, formulir, dan navigasi. DaisyUI mendukung berbagai tema serta mempermudah pengembangan UI modern tanpa menambah ukuran file CSS secara signifikan \cite{daisyuiDocumentation}.

\subsection{Axios}

Axios adalah \textit{HTTP client library} berbasis JavaScript untuk melakukan permintaan HTTP seperti \textit{GET}, \textit{POST}, \textit{PUT}, dan \textit{DELETE}. Axios mendukung Promise serta menyediakan fitur seperti \textit{interceptors}, pengaturan \textit{timeout}, dan otomatis konversi JSON \cite{axiosDocumentation}.

\subsection{UML}

Unified Modeling Language (UML) merupakan bahasa pemodelan standar dalam pengembangan perangkat lunak berorientasi objek. UML digunakan untuk mendokumentasikan, menspesifikasikan, dan membangun sistem secara terstruktur sehingga proses pengembangan dapat dilakukan secara sistematis \cite{sparx2021}.

\section{Use Case Diagram}

Use Case Diagram merupakan pemodelan yang menggambarkan perilaku sistem informasi yang akan dibuat. Use case digunakan untuk mengidentifikasi fungsi-fungsi dalam sistem serta aktor yang memiliki hak akses terhadap fungsi tersebut \cite{sparx2021}. Simbol-simbol yang digunakan dalam Use Case Diagram ditampilkan pada Tabel~\ref{tab:usecase}.

\begin{table}[H]
\centering
\caption{Simbol Use Case Diagram}
\label{tab:usecase}
\begin{tabular}{|c|p{9cm}|}
\hline
\textbf{Gambar} & \textbf{Keterangan} \\ \hline

\includegraphics[width=2.5cm]{images/usecase/usecase.png} &
Use Case menggambarkan fungsionalitas yang disediakan oleh sistem dalam bentuk unit-unit interaksi antara sistem dan aktor. \\ \hline

\includegraphics[width=2cm]{images/usecase/actor.png} &
Actor merupakan entitas yang berinteraksi dengan sistem untuk menjalankan proses tertentu. \\ \hline

\includegraphics[width=3cm]{images/usecase/association.png} &
Asosiasi menunjukkan hubungan interaksi antara aktor dan use case. \\ \hline

\includegraphics[width=3cm]{images/usecase/association_arrow.png} &
Asosiasi dengan panah terbuka menunjukkan interaksi pasif aktor terhadap sistem. \\ \hline

\includegraphics[width=3cm]{images/usecase/include.png} &
Include menunjukkan bahwa suatu use case memanggil use case lain. \\ \hline

\includegraphics[width=3cm]{images/usecase/extend.png} &
Extend menunjukkan perluasan use case berdasarkan kondisi tertentu. \\ \hline

\end{tabular}
\end{table}

\section{Diagram Aktivitas (Activity Diagram)}

Activity Diagram menggambarkan alur kerja atau aktivitas dari suatu sistem maupun proses bisnis \cite{sparx2021}. Simbol-simbol yang digunakan dalam Activity Diagram ditampilkan pada Tabel~\ref{tab:activity}.

\begin{table}[H]
\centering
\caption{Simbol Activity Diagram}
\label{tab:activity}
\begin{tabular}{|c|p{9cm}|}
\hline
\textbf{Gambar} & \textbf{Keterangan} \\ \hline

\includegraphics[width=1.5cm]{images/activity_diagram/start.png} &
Start Point menandakan awal dari suatu aktivitas atau aliran kerja. \\ \hline

\includegraphics[width=1.5cm]{images/activity_diagram/end.png} &
End Point menandakan akhir dari suatu aktivitas atau proses. \\ \hline

\includegraphics[width=3cm]{images/activity_diagram/activity.png} &
Activities menggambarkan proses atau kegiatan bisnis yang dilakukan dalam sistem. \\ \hline

\includegraphics[width=3cm]{images/activity_diagram/fork.png} &
Fork menunjukkan percabangan proses yang berjalan secara paralel dalam sistem. \\ \hline

\includegraphics[width=2.5cm]{images/activity_diagram/decision.png} &
Decision Point menunjukkan titik pengambilan keputusan (true/false) dalam alur kerja. \\ \hline

\includegraphics[width=3cm]{images/activity_diagram/lane.png} &
Swimlane menunjukkan pembagian tanggung jawab aktor dalam suatu proses. \\ \hline

\end{tabular}
\end{table}

\section{Diagram Urutan (Sequence Diagram)}

Sequence Diagram menggambarkan interaksi antar objek dalam suatu use case dengan memperlihatkan waktu hidup objek serta pesan yang dikirim dan diterima \cite{sparx2021}. Simbol-simbol yang digunakan dalam Sequence Diagram ditampilkan pada Tabel~\ref{tab:sequence}.

\begin{table}[H]
\centering
\caption{Simbol Sequence Diagram}
\label{tab:sequence}
\begin{tabular}{|c|p{9cm}|}
\hline
\textbf{Gambar} & \textbf{Keterangan} \\ \hline

\includegraphics[width=2cm]{images/sequence/entity.png} &
Entity Class merepresentasikan entitas utama dalam sistem dan menjadi dasar struktur basis data. \\ \hline

\includegraphics[width=2cm]{images/sequence/boundary.png} &
Boundary Class berfungsi sebagai antarmuka antara aktor dan sistem. \\ \hline

\includegraphics[width=2cm]{images/sequence/control.png} &
Control Class bertanggung jawab mengatur logika aplikasi dan koordinasi antar objek. \\ \hline

\includegraphics[width=3cm]{images/sequence/message.png} &
Message merepresentasikan komunikasi antar objek dalam sistem. \\ \hline

\includegraphics[width=3cm]{images/sequence/recursive.png} &
Recursive menunjukkan objek mengirim pesan kepada dirinya sendiri. \\ \hline

\includegraphics[width=2cm]{images/sequence/activation.png} &
Activation menunjukkan durasi eksekusi suatu operasi pada objek. \\ \hline

\includegraphics[width=2cm]{images/sequence/lifeline.png} &
Lifeline menunjukkan keberadaan objek selama siklus hidupnya. \\ \hline

\end{tabular}
\end{table}

\section{Deployment Diagram}

Deployment Diagram menunjukkan bagaimana dan di mana sistem diterapkan, termasuk arsitektur eksekusi sistem. Perangkat keras, prosesor, dan lingkungan eksekusi perangkat lunak direpresentasikan sebagai \textit{node}. Diagram ini juga dapat menampilkan relasi \textit{deployment} dan \textit{manifest} antar artefak serta node \cite{sparx2021}.

\begin{table}[H]
\centering
\caption{Simbol Deployment Diagram}
\label{tab:deployment}
\begin{tabular}{|c|p{9cm}|}
\hline
\textbf{Simbol} & \textbf{Keterangan} \\ \hline

\includegraphics[width=3cm]{images/deployment/component.png} &
Komponen merupakan bagian modular dari sistem yang perilakunya ditentukan oleh antarmuka yang disediakan dan dibutuhkan. \\ \hline

\includegraphics[width=3cm]{images/deployment/module.png} &
Komponen perangkat lunak yang merepresentasikan unit dengan fungsi tertentu dalam struktur sistem. \\ \hline

\includegraphics[width=3cm]{images/deployment/node.png} &
Node merepresentasikan perangkat keras atau perangkat lunak eksternal dalam sistem. \\ \hline

\includegraphics[width=3cm]{images/deployment/package.png} &
Package adalah wadah untuk mengelompokkan elemen-elemen sistem dalam struktur terorganisir. \\ \hline

\includegraphics[width=3cm]{images/deployment/dependency.png} &
Kebergantungan menunjukkan hubungan antar node, dengan arah panah menuju node yang digunakan. \\ \hline

\includegraphics[width=3cm]{images/deployment/relation.png} &
Relasi menunjukkan hubungan antar node dalam menjalankan sistem. \\ \hline

\end{tabular}
\end{table}
