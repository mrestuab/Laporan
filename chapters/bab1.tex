\chapter{Pendahuluan}

\section{Latar Belakang}

Keterbatasan akses terhadap alat bantu medis di Indonesia masih menjadi masalah kesehatan masyarakat yang signifikan. Banyak penyandang disabilitas dan pasien dengan kebutuhan medis khusus mengalami kesulitan mendapatkan alat bantu seperti kursi roda, tabung oksigen, dan alat rehabilitasi lainnya karena faktor ekonomi dan keterbatasan ketersediaan alat di fasilitas kesehatan. Akses yang terbatas ini tidak hanya memperburuk kondisi kesehatan pengguna tetapi juga berdampak pada tingkat kecacatan dan risiko kematian akibat tidak terpenuhinya kebutuhan dasar medis \cite{pudjiastuti2022}.

Banyak fasilitas kesehatan, terutama rumah sakit umum, sering kali kehabisan stok alat bantu medis yang dibutuhkan oleh pasien, sementara kemampuan masyarakat untuk membeli alat tersebut secara mandiri masih rendah. Hal ini menunjukkan perlunya solusi alternatif agar alat bantu medis dapat diakses oleh masyarakat tanpa harus membeli dengan biaya tinggi \cite{fernandes2022}.

Dalam konteks tersebut, pengembangan sistem peminjaman alat bantu medis menjadi solusi inovatif untuk mengatasi keterbatasan akses tersebut. Sistem ini memungkinkan masyarakat memperoleh alat bantu medis secara temporer melalui mekanisme peminjaman, terutama bagi mereka yang belum mampu membeli alat tersebut atau ketika stok di rumah sakit habis. Dengan adanya sistem ini, diharapkan ketersediaan alat bantu medis bagi masyarakat dapat meningkat serta risiko komplikasi kesehatan akibat keterbatasan alat dapat dikurangi.

\section{Deskripsi Aplikasi}

Aplikasi \textit{MedisLink} merupakan platform berbasis web yang dirancang untuk membantu masyarakat memperoleh akses alat bantu medis secara lebih mudah, cepat, dan efisien melalui sistem peminjaman. Aplikasi ini menjadi solusi bagi individu yang membutuhkan alat bantu medis seperti kursi roda, tongkat kruk, \textit{walker}, atau alat pernapasan namun terkendala biaya atau ketersediaan alat di fasilitas kesehatan.

Melalui MedisLink, pengguna dapat melihat ketersediaan alat, melakukan peminjaman, melacak status pinjaman, serta menerima notifikasi pengembalian secara otomatis. Selain itu, sistem juga mempermudah pengelolaan inventaris bagi admin sehingga pemantauan stok dapat dilakukan secara transparan dan akurat.

Fitur utama aplikasi MedisLink meliputi:

\begin{enumerate}
    \item \textbf{Peminjaman Alat Bantu Medis} \\
    Pengguna dapat memilih jenis alat yang tersedia, melihat detail alat, dan melakukan peminjaman secara daring. Setiap transaksi tersimpan dalam sistem sehingga status peminjaman dapat dipantau.

    \item \textbf{Pengelolaan Inventaris Alat} \\
    Admin dapat mengelola stok alat, menambahkan alat baru, memperbarui kondisi alat, serta mencatat riwayat keluar-masuk alat melalui fitur \textit{Inventory Log}.

    \item \textbf{Notifikasi Pengembalian} \\
    Sistem mengirimkan notifikasi kepada pengguna untuk mengingatkan jadwal pengembalian alat agar dapat digunakan kembali oleh peminjam lainnya.
\end{enumerate}

\section{Identifikasi Masalah}

Berdasarkan latar belakang tersebut, permasalahan yang diidentifikasi adalah:

\begin{enumerate}
    \item Bagaimana membangun sistem peminjaman alat bantu medis yang memudahkan masyarakat mengakses alat tanpa harus membelinya?
    \item Bagaimana sistem dapat membantu mengatasi keterbatasan stok alat bantu medis di rumah sakit atau fasilitas kesehatan?
    \item Bagaimana memfasilitasi masyarakat yang memiliki alat bantu medis tidak terpakai agar dapat mendonasikannya secara tepat sasaran?
\end{enumerate}

\section{Tujuan}

Tujuan pengembangan sistem ini adalah menyediakan solusi berupa sistem peminjaman alat bantu medis yang dapat memperluas akses masyarakat terhadap alat kesehatan.

Tujuan khusus proyek ini adalah:

\begin{enumerate}
    \item Merancang dan membangun platform peminjaman alat bantu medis berbasis web.
    \item Mengembangkan sistem manajemen inventaris untuk melacak perputaran dan ketersediaan alat.
    \item Menyediakan fitur donasi yang memfasilitasi penyaluran alat bantu medis yang tidak terpakai.
\end{enumerate}

\section{Ruang Lingkup}

\subsection{Pengembangan Teknologi}

Dokumentasi ini membahas proses pengembangan sistem peminjaman alat bantu medis berbasis web menggunakan Golang (Fiber) sebagai \textit{backend}, React.js sebagai \textit{frontend}, dan MongoDB sebagai basis data. Sistem dilengkapi dengan autentikasi JWT untuk menjaga keamanan akses pengguna serta struktur API yang efisien \cite{fiber}.

Selain itu, sistem menyediakan fitur notifikasi pengembalian alat untuk mengingatkan pengguna mengenai batas waktu pengembalian. Dokumentasi ini mencakup perancangan arsitektur, desain basis data, serta implementasi fitur aplikasi.

\subsection{Peningkatan Akses dan Efisiensi Layanan}

Melalui pengembangan sistem MedisLink, aplikasi ini diharapkan dapat meningkatkan akses masyarakat terhadap alat bantu medis. Dengan fitur peminjaman daring, pengguna tidak perlu datang langsung ke fasilitas kesehatan sehingga proses menjadi lebih cepat dan efisien. Sistem juga dilengkapi dengan kategori alat, ulasan pengguna, serta pengelolaan inventaris yang terstruktur.
