\section{Latar Belakang}

Keterbatasan akses terhadap alat bantu medis di Indonesia masih menjadi masalah kesehatan masyarakat yang signifikan. Banyak penyandang disabilitas dan pasien dengan kebutuhan medis khusus mengalami kesulitan mendapatkan alat bantu seperti kursi roda, tongkat oksigen, dan alat rehabilitasi lainnya karena faktor ekonomi dan ketersediaan alat di fasilitas kesehatan yang terbatas. Akses yang terbatas ini tidak hanya memperburuk kondisi kesehatan pengguna tetapi juga berdampak pada tingkat kemandirian dan risiko kemalangan akibat tidak tersedianya kebutuhan dasar medis.

Banyak fasilitas kesehatan, terutama rumah sakit umum, sering kali kekurangan stok alat bantu medis yang dibutuhkan oleh pasien, sementara kemampuan masyarakat untuk membeli alat tersebut secara mandiri masih rendah. Hal ini memperlihatkan kebutuhan untuk solusi alternatif agar alat bantu medis dapat diakses oleh masyarakat yang membutuhkan tanpa harus membeli dengan harga tinggi.

Dalam konteks tersebut, pengembangan sistem peminjaman alat bantu medis menjadi solusi inovatif yang dapat membantu mengatasi masalah keterbatasan akses ini. Sistem ini akan memungkinkan masyarakat memperoleh alat bantu medis secara temporal melalui peminjaman, terutama bagi mereka yang tidak mampu membeli alat tersebut dan ketika stok alat di rumah sakit habis. Dengan adanya sistem ini, diharapkan dapat meningkatkan ketersediaan alat bantu medis bagi masyarakat yang membutuhkan dan mengurangi risiko komplikasi kesehatan yang dapat timbul akibat tidak tersedianya alat bantu yang penting.

Berdasarkan kondisi tersebut, aplikasi \textit{MedisLink} dikembangkan sebagai sistem informasi berbasis \textit{web} yang menghubungkan peminjaman dan donasi alat medis. Sistem ini mengintegrasikan teknologi modern seperti \textit{React.js} di sisi depan, \textit{Golang} (\textit{Fiber}) di sisi \textit{backend}, dan \textit{MongoDB} sebagai basis data non-relasional. \textit{MedisLink} dirancang untuk memberikan pengalaman pengguna yang transparan, efisien, dan terpercaya dalam mengakses alat bantu medis.

\section{Identifikasi Masalah}

Meskipun kebutuhan alat bantu medis sangat tinggi di masyarakat, sistem peminjaman alat medis yang terorganisir dan mudah diakses masih terbatas. Sebagian besar fasilitas kesehatan belum memiliki sistem yang dapat memfasilitasi peminjaman alat medis kepada masyarakat umum secara efisien.

Berdasarkan kondisi tersebut, kebutuhan yang muncul adalah:

\begin{enumerate}
    \item Kebutuhan untuk memahami sekaligus mempraktikkan implementasi sistem informasi berbasis \textit{web} untuk sektor kesehatan melalui proyek nyata.
    \item Kebutuhan akan platform peminjaman alat medis yang mudah diakses dan sederhana, tanpa kompleksitas berlebih.
    \item Kebutuhan akan sebuah media inovasi teknologi di bidang kesehatan yang tetap memiliki potensi digunakan oleh publik.
\end{enumerate}

\section{Tujuan}

Tujuan utama dari pengembangan sistem \textit{MedisLink} adalah membangun platform peminjaman alat bantu medis yang sederhana, fungsional, dan dapat menjadi dasar pengembangan lebih lanjut. Secara khusus, tujuan proyek ini adalah:

\begin{enumerate}
    \item Merancang dan mengimplementasikan sistem peminjaman alat medis berbasis \textit{web} sebagai penerapan teknologi informasi di bidang kesehatan.
    \item Menyediakan alternatif sistem peminjaman alat medis yang mudah diakses, fleksibel, dan mudah dikembangkan sesuai kebutuhan.
    \item Membangun pondasi produk digital yang dapat diekspansi menjadi sistem yang lebih kompleks di masa depan sekaligus menjadi sarana pembelajaran dan inovasi di bidang teknologi kesehatan.
\end{enumerate}

\section{Ruang Lingkup}

Ruang lingkup pengembangan ini difokuskan pada perancangan, pembangunan, dan evaluasi prototipe \textit{MedisLink}, yaitu sistem informasi \textit{web} untuk peminjaman alat bantu medis. Lingkup ini mencakup proses analisis kebutuhan, perancangan arsitektur dan model data, implementasi modul inti, serta pengujian fungsional dalam lingkungan pengembangan.

\subsection{Cakupan Fungsional}

Penelitian ini mencakup pengembangan fitur inti sebagai berikut:

\begin{enumerate}
    \item \textbf{Manajemen Pengguna} \\
    Meliputi registrasi, autentikasi, dan otorisasi pengguna menggunakan dua metode:
    \begin{itemize}
        \item \textit{JSON Web Token} (\textit{JWT}) untuk autentikasi berbasis \textit{username/password}.
        \item Sistem verifikasi identitas untuk memastikan keamanan peminjaman alat medis.
    \end{itemize}

    \item \textbf{Pengelolaan Inventaris Alat Medis} \\
    Pencatatan alat medis yang tersedia (nama, kondisi, ketersediaan), penyimpanan data inventaris, serta penyajian informasi ketersediaan alat kepada pengguna.

    \item \textbf{Sistem Peminjaman} \\
    Proses peminjaman alat medis dengan pencatatan identitas peminjam, durasi peminjaman, dan status pengembalian alat.

    \item \textbf{Notifikasi Otomatis} \\
    Sistem notifikasi pengembalian alat otomatis menggunakan mekanisme \textit{Cron Job} dengan interval waktu harian untuk memastikan waktu distribusi inventaris yang tepat.

    \item \textbf{Antarmuka Pengguna \textit{Web}} \\
    Halaman publik untuk menampilkan katalog alat medis dan melakukan proses peminjaman, serta halaman \textit{admin} untuk pengelolaan inventaris.
\end{enumerate}

\subsection{Cakupan Teknis}

Secara teknis, penelitian ini mencakup:

\begin{enumerate}
    \item \textbf{Arsitektur Aplikasi} \\
    Pembangunan sistem dengan arsitektur terpisah menggunakan \textit{React.js} di sisi \textit{frontend}, \textit{Golang} (\textit{Fiber}) di sisi \textit{backend}, dan \textit{MongoDB} sebagai basis data non-relasional.

    \item \textbf{Keamanan Dasar Sistem} \\
    Meliputi:
    \begin{itemize}
        \item Implementasi \textit{JWT} untuk autentikasi pengguna
        \item Validasi input dan sanitasi untuk keamanan data
        \item Penanganan akses \textit{endpoint} privat berdasarkan \textit{token}
        \item Verifikasi identitas untuk peminjaman alat medis
    \end{itemize}

    \item \textbf{Perancangan dan Pemodelan} \\
    Meliputi \textit{use case diagram}, \textit{flowchart} proses (alur peminjaman, autentikasi, inventaris), dan rancangan model data (User, Equipment, Transaction).

    \item \textbf{Integrasi Modul Internal} \\
    Modul koneksi basis data, \textit{middleware} verifikasi \textit{token}, utilitas \textit{hashing}, serta pengelolaan data melalui \textit{schema} User dan Equipment.
\end{enumerate}

\subsection{Cakupan Pengujian}

Pengujian dilakukan meliputi:

\begin{enumerate}
    \item Uji fungsional terhadap \textit{endpoint} inti seperti autentikasi pengguna, pencatatan/pengambilan data peminjaman, dan pemuatan data inventaris.
    \item Uji integritas sederhana untuk memastikan alur peminjaman hingga pengembalian berjalan \textit{end-to-end} menggunakan data simulasi.
\end{enumerate}

\subsection{Batasan Penelitian}

Penelitian ini dibatasi pada aspek-aspek berikut:

\begin{enumerate}
    \item \textbf{Fokus Pada Alur Peminjaman Dasar} \\
    Sistem hanya mencakup proses peminjaman sederhana tanpa fitur pendukung seperti manajemen \textit{maintenance} alat, penjadwalan peminjaman lanjutan, atau sistem rating.

    \item \textbf{Lingkup Pengembangan Berskala Proyek} \\
    Optimasi performa produksi, skalabilitas tinggi, dan \textit{security hardening} tingkat lanjut tidak menjadi fokus utama.

    \item \textbf{Integrasi Sistem Kesehatan} \\
    Sistem tidak terintegrasi dengan sistem informasi rumah sakit atau fasilitas kesehatan lainnya; beroperasi sebagai sistem mandiri untuk tujuan pembelajaran dan pengujian konsep.
\end{enumerate}

\subsection{Luaran Penelitian}

Luaran yang dihasilkan meliputi:

\begin{enumerate}
    \item Prototipe aplikasi \textit{web} \textit{MedisLink} yang dapat dijalankan pada lingkungan pengembangan.
    \item Dokumen desain arsitektur, \textit{use case}, \textit{flowchart}, dan \textit{schema} model data.
    \item Spesifikasi \textit{endpoint API} inti, termasuk manajemen inventaris dan sistem notifikasi.
    \item Hasil pengujian fungsional serta evaluasi ketercapaian kebutuhan.
\end{enumerate}

\section{Sistematika Penulisan}

Sistematika penulisan analisis ini disusun dalam 5 bab dan bagian akhir terdapat daftar pustaka dan lampiran. Di mana pada setiap bab tersebut akan dibagi lagi menjadi sub-bab yang akan dibahas secara terperinci. Berikut merupakan sistematika penulisan dan keterangan singkatnya:

\begin{enumerate}
    \item \textbf{Bab I Pendahuluan} \\
    Berisi latar belakang, identifikasi masalah, tujuan, ruang lingkup, batasan, dan sistematika penulisan. Bab ini juga menjelaskan alasan pemilihan teknologi seperti \textit{React.js}, \textit{Golang} (\textit{Fiber}), \textit{MongoDB}, dan \textit{JWT}.

    \item \textbf{Bab II Landasan Teori} \\
    Menguraikan teori dan konsep pendukung seperti arsitektur \textit{web} modern, sistem informasi kesehatan, autentikasi \textit{JWT}, \textit{API REST}, basis data \textit{NoSQL}, serta \textit{UML}.

    \item \textbf{Bab III Metode Penelitian} \\
    Menjelaskan metode analisis kebutuhan, perancangan, pemodelan, dan pengujian untuk sistem peminjaman alat medis.

    \item \textbf{Bab IV Hasil dan Pembahasan} \\
    Menyajikan hasil implementasi prototipe, struktur proyek, model data, \textit{endpoint}, antarmuka pengguna, hasil pengujian, serta evaluasi sesuai ruang lingkup.

    \item \textbf{Bab V Kesimpulan dan Saran} \\
    Berisi kesimpulan akhir dan rekomendasi pengembangan lebih lanjut sistem \textit{MedisLink}.
\end{enumerate}
